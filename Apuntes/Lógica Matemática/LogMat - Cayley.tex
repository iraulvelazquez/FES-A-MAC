\documentclass{article} % Document type 
\usepackage[spanish]{babel} % Language typesetting 
\usepackage[utf8]{inputenc} % Accented chars without commands
\usepackage[T1]{fontenc} % Accented display corrected
\selectlanguage{spanish} % Default lang, useful for for multi lang switching 
\usepackage{amsmath} % Math 1/3
\usepackage{amssymb} % Math 2/3
\usepackage{stmaryrd} % Math 3/3
\usepackage{tikz} % Crating graphics programmatically
\usepackage{pgfplots} % Extends tikz for 2D and 3D plots for data and math f 
\pgfplotsset{compat=1.18} % ---
\usepackage{circuitikz} % Extends tikz for electrical circuits
\usepackage[dvipsnames, svgnames]{xcolor} % Coloring
\usepackage{soul} % Text highlighting 
\setul{0.5ex}{0.3ex} % ---
\setlength{\parindent}{15pt} % Controls indentantion for the first line of paragraphs 
\setlength{\parskip}{0.8\baselineskip plus 0.2\baselineskip minus 0.1\baselineskip} % Vertical spacing between paragraphs 
\usepackage{geometry} % Layout
\geometry{margin=3cm} % --- 
\usepackage{titlesec} % Global section centering
\titleformat{\section} % --
  {\normalfont\Large\bfseries\centering}{\thesection}{1em}{} % --
\usepackage{changepage} % Special paragraph
\newenvironment{x}[3]{
  \begin{adjustwidth}{#1}{0cm}
  \setlength{\parindent}{0pt}
  #2\fontsize{#3}{1pt}\selectfont 
}{
  \end{adjustwidth}
}

\title{Tablas de Cayley}
\date{06-11-2025}
\author{Velázquez Ramírez Carlos Raúl}

\begin{document}
\pagenumbering{gobble}
  \maketitle
  \newpage
  \pagenumbering{roman}

\section*{Tablas de Cayley}

Para un conjunto finito de elementos, una operación binaria en este puede ser definido por medio de una tabla cuyos elementos son enlistados tanto en la columna como fila principal. Y los resultados de esta operación son presentados en el contenido de la tabla. 

\textbf{Exampli gratia:} Tabla de Cayley para la operación binaria suma en el conjunto $\{2, 3, 5, 7, 11\}$.

\[
  (+, \{2, 3, 5, 7, 11\})
\]
\begin{center}
  \begin{tabular}{|c|c|c|c|c|c|c|c|c|}

  +  & 2  & 3  & 5  & 7  & 11 \\
  2  & 4  & 5  & 7  & 9  & 13 \\ 
  3  & 5  & 6  & 8  & 10 & 14 \\
  5  & 7  & 8  & 10 & 5  & 16 \\
  7  & 9  & 10 & 12 & 6  & 18 \\
  11 & 13 & 14 & 16 & 18 & 22 \\

\end{tabular}
\end{center}

\section*{Operación módulo}

Al operar con los conjuntos $\mathbb{N}$ o $\mathbb{Z}$, una de las restricciones al operar con la división es no poder representar el residuo adecuadamente. Para esto existe la operación módulo, donde el módulo, representado como $mod \, n$, resulta en el residuo de la división impropia $\frac{m}{n}$.

\textbf{Exampli gratia:} Tabla de Cayley del módulo en el conjunto $\{2, 3, 5, 7\}$.

\[
  (mod \, n, n \in \{2, 3, 5, 7\})
\]
\begin{center}
  \begin{tabular}{|c|c|c|c|c|c|c|c|}

  $mod \, n$  & 2  & 3  & 5  & 7  \\
  2           & 0  & 2  & 2  & 2  \\ 
  3           & 1  & 0  & 3  & 10 \\
  5           & 1  & 2  & 0 & 5  \\
  7           & 1  & 1 & 2 & 0  \\

\end{tabular}
\end{center}

\end{document}
