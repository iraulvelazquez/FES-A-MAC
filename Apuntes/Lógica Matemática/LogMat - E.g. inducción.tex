\documentclass{article} % Document type 
\usepackage[spanish]{babel} % Language typesetting 
\usepackage[utf8]{inputenc} % Accented chars without commands
\usepackage[T1]{fontenc} % Accented display corrected
\selectlanguage{spanish} % Default lang, useful for for multi lang switching 
\usepackage{amsmath} % Math 1/3
\usepackage{amssymb} % Math 2/3
\usepackage{stmaryrd} % Math 3/3
\usepackage{tikz} % Crating graphics programmatically
\usepackage{pgfplots} % Extends tikz for 2D and 3D plots for data and math f 
\pgfplotsset{compat=1.18} % ---
\usepackage{circuitikz} % Extends tikz for electrical circuits
\usepackage[dvipsnames, svgnames]{xcolor} % Coloring
\usepackage{soul} % Text highlighting 
\setul{0.5ex}{0.3ex} % ---
\setlength{\parindent}{15pt} % Controls indentantion for the first line of paragraphs 
\setlength{\parskip}{0.8\baselineskip plus 0.2\baselineskip minus 0.1\baselineskip} % Vertical spacing between paragraphs 
\usepackage{geometry} % Layout
\geometry{margin=3cm} % --- 

\title{Inducción}
\date{04-11-2025}
\author{Velázquez Ramírez Carlos Raúl}

\begin{document}
\pagenumbering{gobble}
  \maketitle
  \newpage
  \pagenumbering{roman}

\textbf{Exampli gratia:} 

1) Pruebe para todo número $n \in \mathbb{N}$: 

\[
  1^2 + 2^2 + \dots + n^2 = \frac{n (n + 1)(2n + 1)}{6}
\]

Paso 1: $n = 1$. 

\[
  1 = \frac{1 (1 + 1) (2(1) + 1)}{6} = \frac{6}{6} = 1
\]

Paso 2 (\textit{inductivo}): Suponemos válido para $n = k$. 

\[
  1^2 + 2^2 + \dots + k^2 = \frac{k (k + 1)(2k + 1)}{6}
\]

Y a partir de esto establecemos la validez de la formula para $n = k + 1$.

\[
  1^2 + 2^2 + \dots + (k + 1)^2 = [1^2 + 2^2 + \dots + k^2] + (k + 1)^2
\]

\[
  = \frac{k (k + 1)(2k + 1)}{6} + (k + 1)^2
\]

\[
  = \frac{[k (k + 1) (2k + 1) + 6 (k + 1)^2]}{6} = \frac{(k + 1)[k (2k + 1) + (k + 1)]}{6} = \frac{(k + 1) (2k^2 + 7k + 6)}{6}
\]

\[
  = \frac{(k + 1) (k + 2) (2k + 3)}{6} = \frac{(k + 1)[(k + 1) + 1][2(k + 1) + 1]}{6}
\]

2) Pruebe que para todo número impar $n$, el número $n^2 - 1$ es divisible entre $8$. 

Paso 1 (\textit{de inducción}): $n = 1$. 

\[
  1^2 - 1 = 1 - 1 = 0 = 0 \, (mod8)
\]

Paso 2 (\textit{inductivo}): Suponemos que la propiedad es válida para un número impar positivo cualquiera con $n = 2k + 1$, donde $k \in \mathbb{N} \cup \{0\}$.

\textit{Hipótesis de inducción:} $(2k + 1)^2 - 1 = 0 \, mod8$.

\[
  + \, mod \, 8
\]
\begin{center}
  \begin{tabular}{|c|c|c|c|c|c|c|c|c|}

  + & 0 & 1 & 2 & 3 & 4 & 5 & 6 & 7 \\
  0 & 0 & 1 & 2 & 3 & 4 & 5 & 6 & 7 \\
  1 & 1 & 2 & 3 & 4 & 5 & 6 & 7 & 0 \\
  2 & 2 & 3 & 4 & 5 & 6 & 7 & 0 & 1 \\
  3 & 3 & 4 & 5 & 6 & 7 & 0 & 1 & 2 \\
  4 & 4 & 5 & 6 & 7 & 0 & 1 & 2 & 3 \\
  5 & 5 & 6 & 7 & 0 & 1 & 2 & 3 & 4 \\
  6 & 6 & 7 & 0 & 1 & 2 & 3 & 4 & 5 \\
  7 & 7 & 0 & 1 & 2 & 3 & 4 & 5 & 6 \\
   
\end{tabular}
\end{center}

\[
  * \, mod \, 8
\]
\begin{center}
  \begin{tabular}{|c|c|c|c|c|c|c|c|c|}

  * & 0 & 1 & 2 & 3 & 4 & 5 & 6 & 7 \\
  0 & 0 & 0 & 0 & 0 & 0 & 0 & 0 & 0 \\
  1 & 0 & 1 & 2 & 3 & 4 & 5 & 6 & 7 \\
  2 & 0 & 2 & 4 & 6 & 0 & 2 & 4 & 6 \\
  3 & 0 & 3 & 6 & 1 & 4 & 7 & 2 & 5 \\
  4 & 0 & 4 & 0 & 4 & 0 & 4 & 0 & 4 \\
  5 & 0 & 5 & 2 & 7 & 4 & 1 & 6 & 3 \\
  6 & 0 & 6 & 4 & 2 & 0 & 6 & 4 & 2 \\
  7 & 0 & 7 & 6 & 5 & 4 & 3 & 2 & 1 \\
   
\end{tabular}
\end{center}

Ahora con base en esto, probamos la validez de la propiedad para el siguiente impar. 

\[
  n = 2k + 3
\]

\[
  (2k + 3) - 1 = [(2k + 1)^2 - 1] + 2]^2 - 1 
\]

\[
  (2k + 1)^2 + 2 (2) (2k + 1) + 4 - 1 = (2k + 1)^2 - 1 + 4 (2k + 1) + 4 = [(2k + 1)^2 - 1] + 8k + 8
\]

Se aplica la hipótesis de inducción: 

\[
  (2k + 3)^2 - 1 = [(2k + 1)^2 - 1] + 8k + 8 = 0 + 0 + 0 \, mod \, 8
\]

Por lo que se concluye que $n^2 - 1 = 0 \, (mod \, 8)$ para todo número impar.

3) Pruebe que $n! > 3^{n - 2}$ para cada $n \in \mathbb{N}$, con $n \geq b$.

Paso 1 (\textit{base de inducción}): Procedemos a mostrar la validez de la desigualdad para el primer caso $n = 3$. 

\[
  3! = 1 \cdot 2 \cdot 3 = 6 > 3^1 = 3^{3 - 2}
\]

Paso 2: Suponemos que la desigualdad es válida para $n = k$ donde $k \in \mathbb{N}$ arbitrario, tal que $k \geq 3$.

\textit{Hipótesis de inducción:} $k! > 3^{k - 2}$. 

Ahora probemos la validez de la desigualdad para $n = k + 1$, donde $k > 3$.

\[
  (k + 1)! = 1 \cdot 2 \cdot 3 \cdot \dots \cdot K \cdot (k + 1) = 3^{k - 2} (k + 1) > 3^{k - 2} -3 = 3^{k - 1} = 3{(k + 1) - 2}
\]

$n! > 3^{n - 2}$ para todo natural $n > 3$.

\end{document}
