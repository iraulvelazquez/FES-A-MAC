\documentclass{article} % Document type 
\usepackage[spanish]{babel} % Language typesetting 
\usepackage[utf8]{inputenc} % Accented chars without commands
\usepackage[T1]{fontenc} % Accented display corrected
\selectlanguage{spanish} % Default lang, useful for for multi lang switching 
\usepackage{amsmath} % Math 1/3
\usepackage{amssymb} % Math 2/3
\usepackage{stmaryrd} % Math 3/3
\usepackage{tikz} % Crating graphics programmatically
\usepackage{pgfplots} % Extends tikz for 2D and 3D plots for data and math f 
\pgfplotsset{compat=1.18} % ---
\usepackage{circuitikz} % Extends tikz for electrical circuits
\usepackage[dvipsnames, svgnames]{xcolor} % Coloring
\usepackage{soul} % Text highlighting 
\setul{0.5ex}{0.3ex} % ---
\setlength{\parindent}{15pt} % Controls indentantion for the first line of paragraphs 
\setlength{\parskip}{0.8\baselineskip plus 0.2\baselineskip minus 0.1\baselineskip} % Vertical spacing between paragraphs 
\usepackage{geometry} % Layout
\geometry{margin=3cm} % --- 
\usepackage{titlesec} % Global section centering
\titleformat{\section} % --
  {\normalfont\Large\bfseries\centering}{\thesection}{1em}{} % --
\usepackage{changepage} % Special paragraph
\newenvironment{x}[3]{
  \begin{adjustwidth}{#1}{0cm}
  \setlength{\parindent}{0pt}
  #2\fontsize{#3}{1pt}\selectfont 
}{
  \end{adjustwidth}
}

\title{Lógica Matemática}
\date{09-10-2025}
\author{Velázquez Ramírez Carlos Raúl}

\begin{document}
\pagenumbering{gobble}
  \maketitle
  \newpage
  \pagenumbering{roman}

\section*{Cuantificadores}
\subsection*{E.g.}

Todo estudiante en esta clase ha estudiado precálculo. 

Para todo estudiante x en esta clase, x ha estudiado precálculo. 

$P(x)=$ ha tomado precálculo.

$\forall x Px$

$\forall (x) | P(x)$ 

\subsection*{E.g.}
Todos los estudiantes de clase son varones. 

Para todo estudiante $x$ de la clase, $x$ es varón.

Si $Q(x):x$ es varón entonces se escribe: $\forall x | Q(x)$ 

Si $Q(Adela)=Falso$; si $Q(Efrain)=Verdadero$.

\subsection*{E.g.}

Carlos juega $J(c)$.\\ Sujeto, predicado

Carlos es estudiante $E(c)$.\\ Sujeto, precicado

Si pablo estudia, entonces $2=4$.\\ $E(p) \to 2=4$\\ $E(p) \to 2=4$

Todo perro no ladra.\\ $\forall (x) P(x) \to \neg L(x)$\\ or $\forall x | Px \to \neg Lx$ 

Cualquier país es libre.\\ $\forall x | P(x) \to L(x)$ 

Todos los árboles son verdes.\\ $\forall x | A(x) \to V(x)$

No es cierto que algunos árboles son verdes.\\ $\neg \exists x [ A(x) \and \neg V(x)]$

Ningún árbol es verde.\\ $\forall x | \neg (A(x) \to V(x))$

No es cierto que algún árbol es verde.\\ $\neg \exists x | A(x) \land V(x)$\\ $\neg \exists x | (A(x) \land V(x))$

Algún árbol no es verde.\\ $\exists x | (A(x) \land \neg V(x))$

No es cierto que todo árbol es verde.\\ $\neg \forall x (A(x) \to V(x))$ 

\section*{En desmostraciones} 

Para desmostraciones se introduce un particularizador y un generalizador. 

En generalizaciones:

\begin{center}
  $\forall (x) P(x)$\\ $P(a)$
\end{center}

Donde $P(a)$ elimina el generalizado. 

En particularizaciones:

\begin{center}
  $\exists x | P(x)$\\ $P(a)$
\end{center}

Donde $P(a)$ elimina el particularizado.

\subsection*{E.g.}

Todo hombre es mortal.\\ Todo mortal es débil.\\ $\to$ Todo hombre es débil.

$\forall (x) | H(x) \to M(x)$\\ $\forall (x) | M(x) \to D(x)$\\ $\to \forall (x) | H(x) \to D(x)$

$H(a) \to M(a)$ EG\\ $M(a) \to D(a)$ EG 

Utilizando $3'$ y $4'$ (silogismo hipotético). 

$\to H(a) \to D(a)$\\ $\forall (x) | H(x) \to D(x)$
\end{document}
