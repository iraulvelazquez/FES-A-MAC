\documentclass{article} % Document type 
\usepackage[spanish]{babel} % Language typesetting 
\usepackage[utf8]{inputenc} % Accented chars without commands
\usepackage[T1]{fontenc} % Accented display corrected
\selectlanguage{spanish} % Default lang, useful for for multi lang switching 
\usepackage{amsmath} % Math 1/3
\usepackage{amssymb} % Math 2/3
\usepackage{stmaryrd} % Math 3/3
\usepackage{tikz} % Crating graphics programmatically
\usepackage{pgfplots} % Extends tikz for 2D and 3D plots for data and math f 
\pgfplotsset{compat=1.18} % ---
\usepackage{circuitikz} % Extends tikz for electrical circuits
\usepackage[dvipsnames, svgnames]{xcolor} % Coloring
\usepackage{soul} % Text highlighting 
\setul{0.5ex}{0.3ex} % ---
\setlength{\parindent}{15pt} % Controls indentantion for the first line of paragraphs 
\setlength{\parskip}{0.8\baselineskip plus 0.2\baselineskip minus 0.1\baselineskip} % Vertical spacing between paragraphs 
\usepackage{geometry} % Layout
\geometry{margin=3cm} % --- 

\title{Demostraciones por Epsilon y Delta}
\date{29-10-2025}
\author{Velázquez Ramírez Carlos Raúl}

\begin{document}
\pagenumbering{gobble}
  \maketitle
  \newpage
  \pagenumbering{roman}

\textbf{Exampli gratia:} 

1) Sea $f(x) = 3x - 1$. Dado $\epsilon = 0.01$, encuentre un $\delta > 0$, tal que si $ 0 < |x - 4| < \delta$, entonces $|f(x) - 11| < \epsilon$. 

\[
  \lim\limits_{x \to 4} (3x - 1) = 11
\]

\[
  a=4
\]

\[
  L=11
\]

\[
  0 < |x - a| < \delta \Rightarrow |f(x) - L| < \epsilon
\]

\[
  x \neq a
\]

Probemos que $\lim\limits_{x \to 4} (3x - 1) = 11$. 

Demostración:

Sea $\epsilon > 0$ arbitrario. 

Si elegimos $\delta = \frac{\epsilon}{3}$. 

Suponemos que $0 < |x - 4| < \delta$. 

Se verifica que, 

\[
  |f(x) - 11| = |3x - 1 - 11| = |3x - 12| = 3 |x - 4| < 3 \delta = 3 (\frac{\epsilon}{3}) = \epsilon
\]

\begin{center}
\begin{tikzpicture}
  \begin{axis}
    [
      axis lines = middle,
      xlabel = $x$,
      ylabel = $y$,
      xmax = 13,
      xmin = -5,
      ymax = 14,
      ymin = -5,
      extra x ticks={4},
      extra x tick labels={$a$},
      extra x tick style={tick label style={below}},
      extra y ticks={11},
      extra y tick labels={$L$},
      extra y tick style={tick label style={left}},
    ]
    \addplot
      [
        domain=-15:15,
        samples=150,
        color=blue,
      ] 
    {3*x-1};
  \addlegendentry{$3x-1$}
    \addplot
      [
       color=black,
       only marks,
       mark=*,
       mark size=2pt,
       forget plot,
      ]
    coordinates {(4,11)};
  \addplot
    [
      domain = -10:10,
      samples = 100,
      color = blue,
      line width = 1pt,
      dashed,
    ]
    coordinates {(4,0)(4,11)};
  \addplot
    [
      domain=-0:4,
      samples=100,
      color=blue,
      line width = 1pt,
      dashed,
    ] 
  {11};
  \end{axis}
\end{tikzpicture}
\end{center}

Si $\epsilon = 0.01 = \frac{1}{100}$, entonces $\delta = \frac{1}{3} \cdot \frac{1}{100} = \frac{1}{300}$. 

2) Demuestre que $\lim\limits_{x \to -2} \sqrt{3x + 22} = 4$. 

Demostración: 

Sea $\epsilon > 0$ arbitrario. 

Si elegimos $\delta = 3 \epsilon$ 

Suponemos que $0 < |x + 2| < \delta$ 

Se verifica que 

\[
  |\sqrt{3x + 22} - 4| \text{, donde } \sqrt{3x + 22} = \frac{(\sqrt{3x + 22})(\sqrt{3x + 22})}{\sqrt{3x + 22}}
\]

\[
  \text{, tenemos que } |\frac{3x + 22 - 16}{\sqrt{3x + 22} + 4}| = |\frac{3x + 6}{\sqrt{3x + 22} + 4}| 
\]

\[
  = 3 |x + 2| \cdot \frac{1}{\sqrt{3x + 22} + 4} \text{, porque $\sqrt{3x + 22} + 4 > 0$}
\]

\[
  < 3 \delta \cdot \frac{1}{\sqrt{3x + 22} + 4} < 3 \delta \text{, porque $\sqrt{3x + 22} + 4 > 1$}
\]

\[
  = 3 (\frac{\epsilon}{3}) = \epsilon
\]

3) Demuestre que $\lim\limits_{x \to 3} 6x^2 = 54$. 

Demostración: 

Sea $\epsilon > 0$ arbitrario, 

Si elegimos $\delta = \{1, \frac{\epsilon}{42}\}$. 

Supongamos que $0 < |x - 3| < \delta \Rightarrow |x - 3| < 1 \iff -1 < x - 3 < 1 = 5 < x+3 < 7 = -7 < x + 3 < 7 = |x + 3| < 7$. 

Se verifica que

\[
  |6x^2 - 54| = |6 (x^2 - 9)| = 6 |x - 3| \cdot |x + 3| < 6 \delta \cdot 7 = 42 \delta = 42 (\frac{\epsilon}{42}) = \epsilon
\]

4) Demuestre que $\lim\limits_{x \to 3} \frac{x + 1}{x + 2} = \frac{4}{5}$.

Demostración: 

Sea $\epsilon > 0$. Para encontrar $\delta > 0$, suponemos $|x - 3| < \delta $ y estimamos $|\frac{x+1}{x+2} - \frac{4}{5}| = |\frac{5x + 5 -4x - 8}{5 (x + 2)}| = \frac{1}{5} |\frac{x-3}{x+2}|$.

Si $\delta = 1$, $|x - 3| < 1 \iff -1 < x - 3 < 1 = 4 < x + 2 < 6 = \frac{1}{4} > \frac{1}{x + 2} > \frac{1}{6}$.

\[
  \frac{1}{6} < \frac{1}{x + 2} < \frac{1}{4} \iff |\frac{1}{x + 2}| < \frac{1}{4} \iff \frac{1}{|x + 2|} < \frac{1}{4}
\]

Entonces, 

\[
  |f(x) - L| = \frac{1}{5} |x - 3| \cdot \frac{1}{|x + 2|}
\]

\[
  < \frac{1}{5} \delta \cdot \frac{1}{4} = \frac{1}{20} \delta = \frac{1}{20} (20 \epsilon) = \epsilon
\]

\[
  \therefore \delta = min \{1,20 \epsilon\}
\]

5) Encuentre el límite.

\[
  \lim\limits_{x \to 1} \frac{\frac{1}{\sqrt{x}} - 1}{x - 1} = \frac{0}{0}
\]

\[
  \lim\limits_{x \to 1} \frac{\frac{1 - \sqrt{x}}{\sqrt{x}}}{\frac{x - 1}{1}} = \lim\limits_{x \to 1} \frac{1 - \sqrt{x}}{\sqrt{x} (x - 1)} = \lim\limits_{x \to 1} \frac{-(\sqrt{x} -1)}{\sqrt{x} (\sqrt{x} - 1)(\sqrt{x} + 1)} = \lim\limits_{x \to 1} \frac{-1}{\sqrt{x}(\sqrt{x} + 1)} = -\frac{1}{2}
\]

6) Demuestre que $\lim\limits_{x \to 3} \frac{x + 1}{x + 2} = \frac{4}{5}$.

Demostración: 

Sea $\epsilon > 0$ arbitrario. Para obtener $\delta > 0$, suponemos que $0 < |x - 3| < \delta$ y estimamos, 

\[
  | \frac{x+1}{x+2} - \frac{4}{5}| = | \frac{5x + 5 - 4x - 8}{5 (x + 2)} | = \frac{1}{5} |\frac{x - 3}{x + 2}| = \frac{1}{5} |x -  3| \cdot \frac{1}{|x + 2|}
\]

Si $\delta = 1$, $|x - 3| < 1 \iff -1 < x - 3 < 1 \iff 4 < x + 2 < 6 \iff \frac{1}{4} > \frac{1}{x + 2} > \frac{1}{6}$.

\end{document}
