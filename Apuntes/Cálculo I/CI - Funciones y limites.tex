\documentclass{article}
\usepackage[spanish]{babel}
\usepackage[utf8]{inputenc}
\usepackage[T1]{fontenc}
\selectlanguage{spanish}
\usepackage{amsmath}
\usepackage{amssymb}
\usepackage{stmaryrd}
\usepackage{tikz}
\usepackage{pgfplots}
\pgfplotsset{compat=1.18}
\usepackage[dvipsnames, svgnames]{xcolor}
\usepackage{soul}
\setul{0.5ex}{0.3ex}
\setlength{\parindent}{0pt}
\setlength{\parskip}{0.8\baselineskip plus 0.2\baselineskip minus 0.1\baselineskip}
\usepackage{geometry}
\geometry{margin=3cm}
\usepackage{titlesec} % Global section centering
\titleformat{\section} % --
  {\normalfont\Large\bfseries\centering}{\thesection}{1em}{} % --

\title{Funciones y límites}
\date{13-10-2025}
\author{Velázquez Ramírez Carlos Raúl}

\begin{document}
\pagenumbering{gobble}
  \maketitle
  \newpage
  \pagenumbering{roman}

\section*{Composición de funciones}

Sean $A, B, C \subseteq \mathbb{R}$ y sea $f: A \to B$ y $g: B \to C$ funciones. Se define la función copuesta $f$ con $g$, denotada por $f \circ g: A \to C$ como $(g \circ f)(x)=g(f(x))\, \forall x \in A$.

\textbf{Exampli gratia:} Sean $f(x) = \sqrt{x}$, $g(x)=x^2$, 

\textit{«q compuesta con f»} $(f \circ g)(x)$ 

\begin{align*} 
  (f \circ g)(x)&=f(g(x))\\
          &=(f(x^2))\\
          &=\sqrt{x}\\ 
          &=|x|, Dom_{f \circ g}=\mathbb{R}
\end{align*}

\textit{«f compuesta con q»} $(g \circ f)(x)$ 

\begin{align*}
  (g \circ f)(x)&=g(f(x))\\
          &=g(\sqrt{x})\\
          &=(\sqrt{x})^2\\
          &=x, Dom_{g \circ f}=\mathbb{R}
\end{align*}

\textbf{Observaciones:}

I. $(g \circ f)(x) \neq (f \circ g)(x)$

II. $Dom_{g \circ f}$ está formado por todos los $x \in Dom_f$ tal que $f(x) \in Dom_g$, es decir 

\begin{center}
  $Dom{g \circ f}=\{x \in Dom_f \, | \, f(x) \in Dom_g\}$
\end{center}

\textbf{Exampli gratia:} Sean $f(x)=\sqrt{x}$, $Dom_f=[0,\infty[$ y $g(x)=x^2$ $Dom_g=\mathbb{R}$
\begin{align*}
  Dom_{f \circ g}&=\{x \in Dom_g | g(x) \in Dom_f\}\\
  Dom_{f \circ g}&=\mathbb{R} && \text{\textit{extendido}}\\
  Dom_{f \circ g}&=\mathbb{R} && \text{\textit{restringido}}
\end{align*}

\section*{Función inversa}

Sea $f: A \subseteq \mathbb{R} \to B \subseteq \mathbb{R}$ y $g: B \subseteq \mathbb{R} \to A \subseteq \mathbb{R}$ tal que $(f \circ g)(x)=(g \circ f)(x) \forall x \in A \cap B$, entonces $f \times g$ son funciones inversas una de la otra.

Consecuentemente, $f$ y $g$ son biyectivas y representamos a $g$ como $f^{-1}$. Ademas

\[
  (f \circ f^{-1})(x)=x
\]

\textbf{Exampli gratia:} Muestre que $y=x^2$ y $y=\sqrt{x}$ son inversas una de la otra. Indique dominios, contradominios e imágenes correspondientes. 

\textit{Solución:} Sean $f(x)=\sqrt{x}$ y $g(x)=x^2$. Por un lado, $(f \circ g)(x)=|x|$, por otro, $(g \circ f)(x)=x$. 

\begin{center}
\begin{tikzpicture}
  \begin{axis}[
    axis lines = left,
    xlabel = $x$,
    ylabel = $y$,
    ]
    \addplot[
      domain=0:10,
      samples=100,
      color=blue,
      ] {sqrt(x)};
    \addlegendentry{$\sqrt{x}$}
  \end{axis}
\end{tikzpicture}
\end{center}

\begin{center}
\begin{tikzpicture}
  \begin{axis}[
    axis lines = left,
    xlabel = $x$,
    ylabel = $y$,
    ]
    \addplot[
      domain=0:10,
      samples=100,
      color=blue,
      ] {x^2};
    \addlegendentry{$x^2$}
  \end{axis}
\end{tikzpicture}
\end{center}

Entonces, $(f \circ g)(x)=f(g(x))=f(x^2)=\sqrt{x^2}=x$ (se puede cancelar la raíz y el cuadrado, pues estamos trabajando únicamente en $\mathbb{R^+}$).

Como $f$ y $g$ son biyectivas de $[0,\infty[ \to [0,\infty[$ y $(f \circ g)(x)=(g \circ f)(x)=x$, por lo tanto $f$ y $g$ son inversas una de la otra.

\begin{align*}
  f(x)&=\sqrt{x}, & f^{-1}(x)&=x^2\\
  g(x)&=x^2, & g^{-1}&=\sqrt{x}
\end{align*}

\textit{Exampli gratia:} Calcule la inversa (de existir) de $f(x)=x^3$

\begin{center}
\begin{tikzpicture}
  \begin{axis}[
    axis lines = middle,
    xlabel = $x$,
    ylabel = $y$,
      xmax = 10,
      xmin = -10,
      ymax = 10,
      ymin = -10,
    ]
    \addplot[
      domain=-10:10,
      samples=100,
      color=blue,
      ] {x^3};
    \addlegendentry{$x^3$}
  \end{axis}
\end{tikzpicture}
\end{center}

$f: \mathbb{R} \to \mathbb{R}, Img_f=\mathbb{R}$, $f$ es biyectiva. 

\begin{align*}
  y&=x^3\\
  x^3&=y\\
  x&=y^{1/3}\\
  x&=\sqrt[3]{y} \to y=\sqrt[3]{x}
\end{align*}

Sea $g(x)=\sqrt[3]{x}$, $g: \mathbb{R} \to \mathbb{R}$, $Img_g=\mathbb{R}$, $g$ es una biyección.

\begin{center}
\begin{tikzpicture}
  \begin{axis}[
    axis lines = middle,
    xlabel = $x$,
    ylabel = $y$,
      xmax = 10,
      xmin = -10,
      ymax = 10,
      ymin = -10,
    ]
    \addplot[
      domain=-10:10,
      samples=100,
      color=blue,
      ] {sign(x) * abs(x)^(1/3)};
    \addlegendentry{$\sqrt[3]{x}$}
  \end{axis}
\end{tikzpicture}
\end{center}

\begin{align*}
  (f \circ g)(x)&=f(g(x))=f(\sqrt[3]{x})=(\sqrt[3]{x})^3=x\\
  (g \circ f)(x)&=g(f(x))=g(x^3)=\sqrt[3]{x^3}=x
\end{align*}

Como $f$ y $g$ son biyectivas de $\mathbb{R} \to \mathbb{R}$ y $(f \circ g)(x) = (g \circ f)(x) = x \Rightarrow f$ y $g$ son inversas. 

\textit{Comprobación:} Trazar en un plano $f(x)$, $f^{-1} (x)$ y $y = x$. Un lado de la recta se refleja en la otra.

\begin{center}
\begin{tikzpicture}
  \begin{axis}[
    axis lines = middle,
    xlabel = $x$,
    ylabel = $y$,
      xmax = 10,
      xmin = -6,
      ymax = 10,
      ymin = -6,
    ]
    \addplot[
      domain=-6:10,
      samples=100,
      color=blue,
      ] {x^3};
    \addlegendentry{$x^3$}
    \addplot[
      domain=-6:10,
      samples=100,
      color=red,
      ] {sign(x) * abs(x)^(1/3)};
    \addlegendentry{$\sqrt[3]{x}$}
    \addplot[
      domain=-6:10,
      samples=100,
      color=green,
      ] {x};
    \addlegendentry{$y = x$}
  \end{axis}
\end{tikzpicture}
\end{center}

\end{document}
