\documentclass{article} % Document type 
\usepackage[spanish]{babel} % Language typesetting 
\usepackage[utf8]{inputenc} % Accented chars without commands
\usepackage[T1]{fontenc} % Accented display corrected
\selectlanguage{spanish} % Default lang, useful for for multi lang switching 
\usepackage{amsmath} % Math 1/3
\usepackage{amssymb} % Math 2/3
\usepackage{stmaryrd} % Math 3/3
\usepackage{tikz} % Crating graphics programmatically
\usepackage{pgfplots} % Extends tikz for 2D and 3D plots for data and math f 
\pgfplotsset{compat=1.18} % ---
\usepackage{circuitikz} % Extends tikz for electrical circuits
\usepackage[dvipsnames, svgnames]{xcolor} % Coloring
\usepackage{soul} % Text highlighting 
\setul{0.5ex}{0.3ex} % ---
\setlength{\parindent}{15pt} % Controls indentantion for the first line of paragraphs 
\setlength{\parskip}{0.8\baselineskip plus 0.2\baselineskip minus 0.1\baselineskip} % Vertical spacing between paragraphs 
\usepackage{geometry} % Layout
\geometry{margin=3cm} % --- 

\title{Algunas derivadas}
\date{03-11-2025}
\author{Velázquez Ramírez Carlos Raúl}

\begin{document}
\pagenumbering{gobble}
  \maketitle
  \newpage
  \pagenumbering{roman}

Sea $y = f(x)$ y un punto conocido $(x_0, f(x_0))$ y otro punto sobre la curva $B = (x, f(x))$.

\begin{center}
\begin{tikzpicture}
  \begin{axis}
    [
      axis lines = left,
      xlabel = $x$,
      ylabel = $y$,
      xmax = 10,
      xmin = 0,
      ymax = 50,
      ymin = 0,
      xtick = \empty,
      ytick = \empty,
      extra x ticks={4, 6},
      extra x tick labels={\small$x$, \small$x_0$},
      extra x tick style={tick label style={below, yshift = -2pt}},
      extra y ticks={16, 36},
      extra y tick labels={\small$f(x)$, \small$f(x_0)$},
      extra y tick style={tick label style={left, yshift = -2pt}},
    ]
    \addplot
      [
        domain=0:9,
        samples=100,
        color=purple,
        dashed,
      ] 
    {10*x - 24};
    \addlegendentry{$\sec$}
    \addplot
      [
        domain=0:9,
        samples=100,
        color=blue,
      ] 
    {x^2};
    \addlegendentry{$x^2$}
    \addplot
      [
       color=black,
       only marks,
       mark=*,
       mark size=2pt,
       forget plot,
      ]
    coordinates {(6,36)};
    \addplot
      [
       color=black,
       only marks,
       mark=*,
       mark size=2pt,
       forget plot,
      ]
    coordinates {(4,16)};
  \addlegendentry{$A$}
  \addplot
    [
      domain = -10:10,
      samples = 100,
      color = red,
      line width = 1pt,
      dashed,
    ]
    coordinates {(6,0)(6,36)};
  \addplot
    [
      domain = -10:10,
      samples = 100,
      color = green,
      line width = 1pt,
      dashed,
    ]
    coordinates {(4,0)(4,16)};
  \addplot
    [
      domain=-0:4,
      samples=100,
      color= green,
      line width = 1pt,
      dashed,
    ] 
  {16};
  \addlegendentry{$B$}
  \addplot
    [
      domain=-0:6,
      samples=100,
      color=red,
      line width = 1pt,
      dashed,
      ] 
      {36};
  \end{axis}
\end{tikzpicture}
\end{center}

La pendiente de la secante que pasa por $A$ y por $B$ en $m_{\sec}$ se define como

\[
  f'(x_0) = \lim\limits_{x \to x_0} \frac{f(x) - f(x_0)}{x - x_0}
\]

Si $B \to A$ entonces $m_{\sec} = m_{\tan}$.

\[
  m_{\tan} = \lim\limits_{x \to x_0} m_{\sec} = \lim\limits_{x \to x_0} \frac{f(x) - f(x_0)}{x - x_0} = f'(x_0)
\]

\[
  \frac{f(x) - f(x_0)}{x - x_0} \text{, cociente diferencial}
\]

\textbf{Exampli gratia:} 

1) Encuentre la ecuación de la recta tangente a la gráfica $y = x^3$ en el punto de la abscisa $x = 0$.

Sea $f(x) = x^3$ y $x_0 = 0$.

Si $x \neq 0$, calculamos el cociente diferencial, 

\[
  \frac{f(x) - f(x_0)}{x - x_0} = \frac{x^3 - 0}{x - 0} = \frac{x^3}{x} = x^2
\]

Entonces, 

\[
  \frac{f(x) - f(x_0)}{x - x_0} = \lim\limits_{x \to 0} x^2 = 0
\]

Así, $f'(0) = 0$.

Calculemos la ecuación de la recta con $m = f'(0) = 0$ y $(0,0)$: 

\[
  y - y_1 = m(x - x_1)
\]

\[
  y = 0
\]

\begin{center}
\begin{tikzpicture}
  \begin{axis}
    [
      axis lines = middle,
      xlabel = $x$,
      ylabel = $y$,
      xmax = 3,
      xmin = -3,
      ymax = 3,
      ymin = -3,
    ]
    \addplot
      [
        domain=-3:3,
        samples=150,
        color=blue,
      ] 
    {x^3};
  \addlegendentry{$x^3$}
    \addplot
      [
        domain=-3:3,
        samples=150,
        color=red,
      ] 
    {0};
  \addlegendentry{$y = 0$}
  \end{axis}
\end{tikzpicture}
\end{center}

2) Demuestre que $f(x) = \sqrt{x}$ es derivable en $]0, \infty[$ pero no lo es en $x = 0$.

Formulación (no recomendable):

\[
  y = \sqrt{x} = x^{\frac{1}{2}}
\]

\[
  y' = \frac{1}{2} x^{-\frac{1}{2}} = \frac{1}{2 \sqrt{x}}
\]

Por definición:

\[
  y = \sqrt{-x^2 - 1} = (-x^2 - 1)^\frac{1}{2}
\]

\[
  y' = \frac{1}{2} (-x^2 - 1)^{-\frac{1}{2}} \cdot (-2x) = - \frac{x}{\sqrt{-x^2 - 1}}
\]

\begin{center}
  \textit{No hay solución en $\mathbb{R}$}
\end{center}

\textit{Solución:} Sea $x_o \in ]0, \infty[$.

Si $x \neq x_0$, entonces, 

\[
  \frac{f(x) - f(x_1)}{x - x_0} = \frac{\sqrt{x} - \sqrt{x_0}}{x - x_0} \cdot \frac{\sqrt{x} + \sqrt{x_0}}{\sqrt{x} + \sqrt{x_0}} = \frac{x - x_0}{(x - x_0)(\sqrt{x} + \sqrt{x_0})} = \frac{1}{\sqrt{x} + \sqrt{x_0}}
\]

Consecuentemente, 

\[
  \lim\limits_{x \to x_0} \frac{f(x) - f(x_0)}{x - x_0} = \lim\limits_{x \to x_0} \frac{1}{\sqrt{x} + \sqrt{x_0}} = \frac{1}{2 \sqrt{x_0}}
\]

Como $x_0 > 0$, $f'(x_0) = \frac{1}{2 \sqrt{x_0}}$ está bien definido.

Pero, si $x_0 = 0$, $\lim\limits_{x \to 0} \frac{f(x) - f(0)}{x - 0} = \lim\limits_{x \to 0} \frac{\sqrt{x} - 0}{x - 0} = \lim\limits_{x \to 0} \frac{\sqrt{x}}{x} = \lim\limits_{x \to 0} \frac{1}{\sqrt{x}} = \infty \therefore f'(x)$ no existe.

3) Demuestre que $f(x) = |x - 2|$ no es derivable en $x_0 = 2$.

\[|x - 2|
  \begin{cases}
    x - 2 & \text{, si } x \geq 2\\
    -(x - 2) & \text{, si } x < 2
  \end{cases}
\]

Debido al punto esquina, el límite no es derivable.

4) Sea $x_0 = 2$ y $x \neq x_0$. Calculemos el coeficiente diferencial.

\[
  \frac{f(x) - f(x_0)}{x - x_0} = \frac{|x - 2| - 0}{x - 2} - \frac{|x - 2|}{x - 2}
\]

\begin{center}
\begin{tikzpicture}
  \begin{axis}
    [
      axis lines = middle,
      xlabel = $x$,
      ylabel = $y$,
      xmax = 6,
      xmin = -3,
      ymax = 6,
      ymin = -1,
    ]
    \addplot
      [
        domain=-6:6,
        samples=150,
        color=blue,
      ] 
    {abs(x-2)};
  \addlegendentry{$x^3$}
  \end{axis}
\end{tikzpicture}
\end{center}

Para calcular $\lim\limits_{x \to 2} \frac{f(x) - f(2)}{x - 2}$, necesitamos estudiar los límites laterales: 

Por izq.:

\[
  \lim\limits_{x \to 2^-} \frac{f(x) - f(2)}{x - 2} = \lim\limits_{x \to 2} \frac{-(x - 2) - 0}{x - 2} = -1 \text{, con $x < 2$}
\]

Por der.:

\[
  \lim\limits_{x \to 2^+} \frac{f(x) - f(2)}{x - 2} = \lim\limits_{x \to 2} \frac{(x - 2) - 0}{x - 2} = 1 \text{, con $x > 2$}
\]

Como los límites laterales son distintos, $\lim\limits_{x \to 2^+} \frac{f(x) - f(2)}{x - 2}$ no existe. 

Por tanto $f$ no es derivable en $x_0 = 2$.

No ha de hacer el siguiente caso, donde: 

\[f'(x)
  \begin{cases}
    1  & \text{, si } x \geq 2\\
    -1 & \text{, si } x < 2
  \end{cases}
\]

5) Determine la derivada de $f(x) = \frac{x + 6}{x - 4}$ en cualquier punto del dominio. 

Sea $x \in \mathbb{R} \setminus \{4\}$. Si $x \neq x_0$, calculamos el cociente diferencial. 

\[
  \frac{f(x) - f(x_0)}{x - x_0} = \frac{\frac{x + 6}{x - 4} - \frac{x_0 + 6}{x_0 - 4}}{x - x_0} = \frac{\frac{(x_0 - 4)(x + 6) - (x - 4)(x_0 + 6)}{(x - 4)(x_0 - 4)}}{\frac{x - x_0}{1}}
\]
\[
  = \frac{(x_0 - 4)(x + 6) - (x - 4)(x_0 + 6)}{(x - 4)(x_0 - 4)(x - x_0)} = \frac{x_0 x + 6x_0 - 4x - 24 - x x_0 - 6x + 4x_0 + 24}{(x - 4)(x_0 - 4)(x - x_0)} 
\]

\[
= \frac{10 (x_0 - x)}{(x - 4)(x_0 - 4)(x - x_0)} = \frac{-10}{(x - 4)(x_0 - 4)}
\]

Entonces, 

\[
  \lim\limits_{x \to x_0} \frac{f(x) - f(x_0)}{x - x_0} = \lim\limits_{x \to x_0} \frac{-10}{(x - 4)(x_0 - 4)} = \frac{-10}{(x_0 -4)^2}
\]

6) Sea $\lim\limits_{x \to -3^+} \frac{|x - 4| - 7}{x + 3}$.

\[ |x - 4|
  \begin{cases}
    x - 4  & \text{, si } x \geq 4\\
    -(x - 4) & \text{, si } x < 4
  \end{cases}
\]

Puesto que $-3 < 4$ entonces se toma el caso dos, 

\[
  \lim\limits_{x \to -3^+} \frac{-(x - 4) - 7}{x + 3} = \lim\limits_{x \to -3^+} \frac{-x - 3}{x + 3} \text{, con $x > -3$}
\]

7) Sea $\lim\limits_{x \to \pi} \frac{\tan(x - \pi)}{x - \pi} = \frac{0}{0}$, indeterminación.

Sea $u = x - \pi$, de tal modo que conforme $x$ se acerce a $\pi$, $u$ se acerca a $0$. Entonces $\frac{\tan(x - \pi)}{x - \pi} = \frac{\tan u}{u}$.

Como $x \to \pi$, entonces $u \to 0$.

Consideremos que, 

\[
  \lim\limits_{x \to 0} \frac{\tan x}{x} = \frac{0}{0} \text{, indeterminación}
\]

\[
  = \lim\limits_{x \to 0} \frac{\frac{\sin x}{\cos x}}{\frac{x}{1}} = \lim\limits_{x \to 0} \frac{\sin x}{x \cos x} = \lim\limits_{x \to 0} \frac{\sin x}{x} \cdot \lim\limits_{x \to 0} \frac{1}{\cos x} = 1 \cdot 1 = 1
\]

Entonces, 

\[
  \lim\limits_{u \to 0} \frac{\tan u}{u} = \lim\limits_{u \to 0} \frac{\frac{\sin u}{\cos u}}{\frac{u}{1}} = \lim\limits_{u \to 0} \frac{\sin u}{u \cos u} = \lim\limits_{u \to 0} \frac{\sin u}{u} \cdot \lim\limits_{u \to 0} \frac{1}{\cos u} = 1 \cdot 1 = 1
\]

\end{document}
