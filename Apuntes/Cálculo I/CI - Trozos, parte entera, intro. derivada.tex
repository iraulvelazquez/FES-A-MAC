\documentclass{article} % Document type 
\usepackage[spanish]{babel} % Language typesetting 
\usepackage[utf8]{inputenc} % Accented chars without commands
\usepackage[T1]{fontenc} % Accented display corrected
\selectlanguage{spanish} % Default lang, useful for for multi lang switching 
\usepackage{amsmath} % Math 1/3
\usepackage{amssymb} % Math 2/3
\usepackage{stmaryrd} % Math 3/3
\usepackage{tikz} % Crating graphics programmatically
\usepackage{pgfplots} % Extends tikz for 2D and 3D plots for data and math f 
\pgfplotsset{compat=1.18} % ---
\usepackage{circuitikz} % Extends tikz for electrical circuits
\usepackage[dvipsnames, svgnames]{xcolor} % Coloring
\usepackage{soul} % Text highlighting 
\setul{0.5ex}{0.3ex} % ---
\setlength{\parindent}{15pt} % Controls indentantion for the first line of paragraphs 
\setlength{\parskip}{0.8\baselineskip plus 0.2\baselineskip minus 0.1\baselineskip} % Vertical spacing between paragraphs 
\usepackage{geometry} % Layout
\geometry{margin=3cm} % --- 
\usepackage{titlesec} % Global section centering
\titleformat{\section} % --
  {\normalfont\Large\bfseries\centering}{\thesection}{1em}{} % --
\usepackage{titlesec} % Global subsection centering
\titleformat{\subsection} % --
  {\normalfont\large\bfseries\centering}{\thesection}{1em}{} % --

\title{Funciones a trozos, función parte entera e introdcción intuitiva a la derivada}
\date{31-10-2025}
\author{Velázquez Ramírez Carlos Raúl}

\begin{document}
\pagenumbering{gobble}
  \maketitle
  \newpage
  \pagenumbering{roman}

\section*{Función a trozos}

\textbf{Exampli gratia:} 

1) 

\[\text{Sea: $f(x)$} 
  \begin{cases}
    x & \text{, si } 0 \leq x < 1\\ 
    3x - 2 & \text{, si } 1 \leq x \leq 2
  \end{cases}
\]

Purebe que $\lim\limits_{x \to 1} f(x)$ existe mediante la definción $\epsilon-\delta$.

Proponemos $\lim\limits_{x \to 1} f(x) = 1$

Caso 1: Sea $0 <= x < 1$ y sea $\epsilon > 0$. 

Si elegimos $\delta_1 = \epsilon$

Suponemos que $0 < |x - 1| < \delta$. 

Verificamos que $|f(x) - L| = |x - 1| < \delta_1 = \epsilon$.

Caso 2: Sea $1 <= x <= 2$ y $\epsilon > 0$. 

Si elegimos $\delta_2 = \frac{\epsilon}{3}$.

Suponemos $0 < |x - 1| < \delta_2$. 

Verificamos que $|f(x) - L| = |3x - 2 - 1| = |3x - 3| = 3 |x - 1| < 3 \delta_2 = 3 (\frac{\epsilon}{3}) = \epsilon$.

Por tanto, 

\[
  \delta = min \{\delta_1, \delta_2\}
\]

\[
  = min \{\epsilon, \frac{\epsilon}{3}\} = \frac{\epsilon}{3}
\]

Siempre se ha de tomar la $\epsilon$ más pequeña. Solo puede existir una, por definción.

2) 

\[\text{Sea: $f(x)$} 
  \begin{cases}
    2x + A & \text{, si } x < 1\\
    x & \text{, si } 1 \leq x \leq 5\\
    x^2 + B & \text{, si } 5 < x
  \end{cases}
\]

Veamos la continuidad en $x = 1$. 

Por un lado, $f(1) = 1$ y $\lim\limits_{x \to 1} f(x)$. 

Se calcula con límites laterakes: 

\[
  \lim\limits_{x \to 1^-} f(x) = \lim\limits_{x \to 1} (2x + A) = 2 + A \text{, donde $x < 1$}
\]

\[
  \lim\limits_{x \to 1^+} f(x) = \lim\limits_{x \to 1} x = 1 \text{, donde $x > 1$}
\]

Entonces, para que $\lim\limits_{x \to 1} f(x)$ existe, debe suceder: 

\[
  2 + A = 1 \Rightarrow A = -1
\]

Así, $f (1) = \lim\limits_{x \to 1} f(x) = 1$, por tanto $f$ es continua en $x = 1$.

Veamos la continuidad de $f$ en $x = 5$. 

Notemos que $f(5) = 5$. 

Para hallar $\lim\limits_{x \to 5} f(x)$, debemos estudiar límites laterales: 

\[
  \lim\limits_{x \to 5^-} f(x) = \lim\limits_{x \to 5^-} x = 5 \text{, donde $x < 5$}
\]

\[
  \lim\limits_{x \to 5^+} f(x) = \lim\limits_{x \to 5^+}  (x^2 + B) = 25 + B \text{, donde $x > 5$}
\]

Para que exista $\lim\limits_{x \to 5} f(x)$, los laterales deben coincidir.

\[
  25 + B = 5 \Rightarrow B = -20
\]

Así $f(5) = \lim\limits_{x \to 5} f(x) = 5$ y, por tanto, $f$ es continua en $x = 5$.

Como $f$ se define a trozos mediante polinomios, $f$ es continua en el resto del dominio.

\begin{center}
\begin{tikzpicture}
  \begin{axis}
    [
      axis lines = middle,
      xlabel = $x$,
      ylabel = $y$,
      xmax = 15,
      xmin = -3,
      ymax = 50,
      ymin = -5,
    ]
    \addplot
      [
        domain=-3:1,
        samples=150,
        color=blue,
      ] 
    {2*x-1};
  \addlegendentry{$2x - 1$}
    \addplot
      [
        domain=1:5,
        samples=150,
        color=green,
      ] 
    {x};
  \addlegendentry{$x$}
    \addplot
      [
        domain=5:10,
        samples=150,
        color=red,
      ] 
    {x^2-20};
  \addlegendentry{$x^2 - 20$}
  \end{axis}
\end{tikzpicture}
\end{center}

\[\text{Sea: $f(x)$} 
  \begin{cases}
    2x - 1 & \text{, si } x < 1\\
    x & \text{, si } 1 \leq x \leq 5\\
    x^2 - 20 & \text{, si } 5 < x
  \end{cases}
\]

\section*{Función parte entera}

\subsection*{Función techo}

\[
E \, : \, \mathbb{R} \to \mathbb{Z}
\]
\[
  x \mapsto y = E(x)
\]

\[
techo \, : \, \mathbb{R} \to \mathbb{Z}
\]
\[
  techo(x) = \lceil x \rceil := min \{ K \in \mathbb{Z} \, | \, x \leq K \}
\]

\[\textbf{E.g.}
  \begin{cases}
    \text{Si $x = 0$} & \lceil 0 \rceil = 0\\
    \text{Si $x = n, \, n \in \mathbb{Z}$} & \lceil n \rceil = n\\
    \text{Si $x = 0.9$} & \lceil 0 \rceil = 1\\
    \text{Si $x = 1.4$} & \lceil 0 \rceil = 2\\
    \text{Si $x = 2$} & \lceil 0 \rceil = 2\\
    \text{Si $x = -3$} & \lceil 0 \rceil = -3
  \end{cases}
\]

\subsection*{Función suelo}

\[
  suelo \, : \, \mathbb{R} \to \mathbb{Z} 
\]
\[
suelo(x) = \lfloor x \rfloor := max \{ k \in \mathbb{Z} \, | \, k \leq x \}
\]

\[\textbf{E.g.}
  \begin{cases}
    \text{Si $x = 0$} & \lceil 0 \rceil = 0\\
    \text{Si $x = n, \, n \in \mathbb{Z}$} & \lceil n \rceil = n\\
    \text{Si $x = 0.9$} & \lceil 0 \rceil = 0\\
    \text{Si $x = 1.4$} & \lceil 0 \rceil = 1\\
    \text{Si $x = 2$} & \lceil 0 \rceil = 2\\
    \text{Si $x = -3$} & \lceil 0 \rceil = -3
  \end{cases}
\]

\begin{center}
\begin{tikzpicture}
  \begin{axis}
    [
      axis lines = middle,
      xlabel = $x$,
      ylabel = $y$,
      xmax = 5,
      xmin = -5,
      ymax = 5,
      ymin = -5,
    ]
    \addplot
      [
        domain=-5:5,
        samples=150,
        color=blue,
      ] 
    {floor(x)};
  \addlegendentry{$\lfloor x \rfloor$}
    \addplot
      [
        domain=-5:5,
        samples=150,
        color=red,
      ] 
    {ceil(x)};
  \addlegendentry{$\lceil x \rceil$}
  \end{axis}
\end{tikzpicture}
\end{center}

\section*{La derivada} 

\textit{Def.} Sean $I$ un intervalo en $\mathbb{R}$ y $f \, : \, I \to \mathbb{R}$. Si $x_0 \in I$, definimos la derivadad de $f$ en $x_0$ como 

\[
  f'(x_0) = \lim\limits_{x \to x_0} \frac{f(x) - f(x_0)}{x - x_0}
\]

, siempre que el límite exista.

Si el límite existe, se dice que $f$ es derivable o diferenciable en el punto $x = x_0$.

\textit{Motivación:} Encontrar la pendiente de una recta tangente a la gráfica de una función en un punto. 

\begin{center}
\begin{tikzpicture}
  \begin{axis}
    [
      axis lines = left,
      xlabel = $x$,
      ylabel = $y$,
      xmax = 10,
      xmin = 0,
      ymax = 50,
      ymin = 0,
      xtick = \empty,
      ytick = \empty,
      extra x ticks={4, 6},
      extra x tick labels={\small$x$, \small$x + h$},
      extra x tick style={tick label style={below, yshift = -2pt}},
      extra y ticks={16, 36},
      extra y tick labels={\small$f(x)$, \small$f(x + h)$},
      extra y tick style={tick label style={left, yshift = -2pt}},
    ]
    \addplot
      [
        domain=0:9,
        samples=100,
        color=purple,
        dashed,
      ] 
    {10*x - 24};
    \addlegendentry{$\sec$}
    \addplot
      [
        domain=0:9,
        samples=100,
        color=blue,
      ] 
    {x^2};
    \addlegendentry{$x^2$}
    \addplot
      [
       color=black,
       only marks,
       mark=*,
       mark size=2pt,
       forget plot,
      ]
    coordinates {(5,25)};
    \addlegendentry{$B$}
    \addplot
      [
       color=black,
       only marks,
       mark=*,
       mark size=2pt,
       forget plot,
      ]
    coordinates {(6,36)};
    \addplot
      [
       color=black,
       only marks,
       mark=*,
       mark size=2pt,
       forget plot,
      ]
    coordinates {(4,16)};
  \addlegendentry{$A$}
  \addplot
    [
      domain = -10:10,
      samples = 100,
      color = red,
      line width = 1pt,
      dashed,
    ]
    coordinates {(6,0)(6,36)};
  \addplot
    [
      domain = -10:10,
      samples = 100,
      color = blue,
      line width = 1pt,
      dashed,
    ]
    coordinates {(5,0)(5,25)};
  \addplot
    [
      domain=-0:5,
      samples=100,
      color=blue,
      line width = 1pt,
      dashed,
    ] 
  {25};
  \addlegendentry{$S$}
  \addplot
    [
      domain = -10:10,
      samples = 100,
      color = green,
      line width = 1pt,
      dashed,
    ]
    coordinates {(4,0)(4,16)};
  \addplot
    [
      domain=-0:4,
      samples=100,
      color= green,
      line width = 1pt,
      dashed,
    ] 
  {16};
  \addplot
    [
      domain=-0:6,
      samples=100,
      color=red,
      line width = 1pt,
      dashed,
    ] 
  {36};
  \end{axis}
\end{tikzpicture}
\end{center}

\[
  A(x, f(x))
\]
\[
  B(x + h, f(x + h))
\]
\[
  m_{\sec} = \frac{f(x + h) - f(x)}{x + h - x} =  \frac{f(x + h) - f(x)}{h}
\]

Movemos el punto $B$ sobre la curva para que se acerque al punto $A$. 

\[
  B \to A \, : \, m_{\tan} = \lim\limits_{h \to 0} m_{\sec} = \lim\limits_{h \to 0} \frac{f(x + h) - f(x)}{h} = f'(x)
\]

\textbf{Exampli gratia:} Calcule la pendiente de la recta tangente a la gráfica  de $y = x^2$ en el punto de la abscisa $x = 5$.

Sea 

\[
  f'(x) = \lim\limits_{h \to 0} \frac{f(x + h) - f(x)}{h} = \lim\limits_{h \to 0} \frac{(x + h)^2 - x^2}{h} = \lim\limits_{h \to 0} \frac{x^2 + 2xh + h^2 - x^2}{h} = \lim\limits_{h \to 0} \frac{h (2x + h)}{h}
\]

\[
  = \lim\limits_{h \to 0} (2x + h) = 2x \Rightarrow f'(x) = 2x \, \forall x \in Dom_f
\]

\[
  f'(5) = 10
\]

Encontremos la ecuación de la recta tangente con $m = 10$ y $(5, 25)$. 

\[
  y - y_1 = m(x - x_1)
\]
\[
  y - 25 = 10(x - 5)
\]
\[
  y = 10x - 25
\]

\begin{center}
\begin{tikzpicture}
  \begin{axis}
    [
      axis lines = middle,
      xlabel = $x$,
      ylabel = $y$,
      xmax = 13,
      xmin = -2,
      ymax = 50,
      ymin = -3,
    ]
    \addplot
      [
        domain=-2:13,
        samples=150,
        color=blue,
      ] 
    {x^2};
  \addlegendentry{$x^2$}
    \addplot
      [
        domain=-2:13,
        samples=150,
        color=red,
      ] 
    {10*x-25};
  \addlegendentry{$10x - 25$}
    \addplot
      [
       color=black,
       only marks,
       mark=*,
       mark size=2pt,
       forget plot,
      ]
    coordinates {(5,25)};
  \end{axis}
\end{tikzpicture}
\end{center}

\end{document}
