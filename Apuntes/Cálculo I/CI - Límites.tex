\documentclass{article} % Document type 
\usepackage[spanish]{babel} % Language typesetting 
\usepackage[utf8]{inputenc} % Accented chars without commands
\usepackage[T1]{fontenc} % Accented display corrected
\selectlanguage{spanish} % Default lang, useful for for multi lang switching 
\usepackage{amsmath} % Math 1/3
\usepackage{amssymb} % Math 2/3
\usepackage{stmaryrd} % Math 3/3
\usepackage{tikz} % Crating graphics programmatically
\usepackage{pgfplots} % Extends tikz for 2D and 3D plots for data and math f 
\pgfplotsset{compat=1.18} % ---
\usepackage{circuitikz} % Extends tikz for electrical circuits
\usepackage[dvipsnames, svgnames]{xcolor} % Coloring
\usepackage{soul} % Text highlighting 
\setul{0.5ex}{0.3ex} % ---
\setlength{\parindent}{15pt} % Controls indentantion for the first line of paragraphs 
\setlength{\parskip}{0.8\baselineskip plus 0.2\baselineskip minus 0.1\baselineskip} % Vertical spacing between paragraphs 
\usepackage{geometry} % Layout
\geometry{margin=3cm} % --- 
\usepackage{titlesec} % Global section centering
\titleformat{\section} % --
  {\normalfont\Large\bfseries\centering}{\thesection}{1em}{} % --

\title{Límites}
\date{15-10-2025}
\author{Velázquez Ramírez Carlos Raúl}

\begin{document}
\pagenumbering{gobble}
  \maketitle
  \newpage
  \pagenumbering{roman}

\section*{Límites y casos}

1) $\lim\limits_{x \to \infty} [f(x) \pm g(x)] = \lim\limits_{x \to a} f(x) \pm \lim\limits_{x \to a} g(x)$ si los limites existen.

2) $\lim\limits_{x \to a} f(x) = c \lim\limits_{x \to a} f(x)$ si los limites existen.

3) $\lim\limits_{x \to a} \frac{f(x)}{g(x)} = \frac{\lim\limits_{x \to a} f(x)}{\lim\limits_{x \to a} g(x)}$ si los limites existen y $g(x) \neq 0$. 

4) $\lim\limits_{x \to a} [f(x)]^n = [\lim\limits_{x \to a} f(x)]^n$ si los limites existen, con $n \in \mathbb{R}$ y $f(x) \neq 0$ y $n \neq 0$. 

\textbf{Exampli gratia:} Calcula los siguientes límites. 

1) $\lim\limits_{x \to 5} x^2$ 

Tendríamos que $\lim\limits_{x \to 5} x^2 = 5^2=25$. 

Gráficamente: 

\begin{center}
\begin{tikzpicture}
  \begin{axis}
    [
      axis lines = middle,
      xlabel = $x$,
      ylabel = $y$,
      xmax = 10,
      xmin = -10,
      ymax = 30,
      ymin = 0,
    ]
    \addplot
      [
        domain=-10:10,
        samples=100,
        color=blue,
      ] 
    {x^2};
    \addlegendentry{$x^2$}
    \addplot
      [
       color=black,
       only marks,
       mark=*,
       mark size=2pt,
       forget plot,
      ]
    coordinates {(5,25)};
  \addplot
    [
      domain = -10:10,
      samples = 100,
      color = blue,
      line width = 1pt,
      dashed,
    ]
    coordinates {(5,0)(5,25)};
  \addplot
    [
      domain=-0:5,
      samples=100,
      color=blue,
      line width = 1pt,
      dashed,
    ] 
  {25};
  \end{axis}
\end{tikzpicture}
\end{center}

2) $\lim\limits_{x \to 1} (x+2)$

Tendríamos que $\lim\limits_{x \to 1} (x+2)=\lim\limits_{x \to 1} x + \lim\limits_{x \to 1} 2 = 1 + 2 = 3$.

Gráficamente: 

\begin{center}
\begin{tikzpicture}
  \begin{axis}
    [
      axis lines = middle,
      xlabel = $x$,
      ylabel = $y$,
      xmax = 5,
      xmin = -5,
      ymax = 5,
      ymin = -5,
    ]
    \addplot
      [
        domain=-5:5,
        samples=100,
        color=blue,
      ] 
    {x+2};
    \addlegendentry{$f(x)=x+2$}
    \addplot
      [
        domain = 0:1,
        samples = 100,
        color = blue,
        line width = 1pt,
        dashed,
      ]
    {3};
    \addplot
      [
        color = black,
        only marks,
        mark=*,
        mark size = 2pt,
        forget plot,
      ]
    coordinates {(1,3)};
    \addplot
      [
        domain = -5:5,
        samples = 100,
        color = blue,
        line width = 1pt,
        dashed,
      ]
      coordinates {(1,0)(1,3)};
  \end{axis}
\end{tikzpicture}
\end{center}

\begin{center}
\begin{tikzpicture}
  \begin{axis}
    [
      axis lines = middle,
      xlabel = $x$,
      ylabel = $y$, 
      xmin = -5,
      xmax = 5,
      ymin = -5,
      ymax = 5,
    ]
    \addplot
      [
        domain=-5:5,
        samples=100,
        color=blue,
      ] 
    {x};
    \addlegendentry{$f(x)=x$}
    \addplot
      [
        color = black,
        only marks,
        mark=*,
        mark size = 2pt,
        forget plot,
      ]
    coordinates {(1,1)};
    \addplot
      [
        domain = -5:5,
        samples = 100,
        color = blue,
        line width = 1pt,
        dashed,
      ]
      coordinates {(1,0)(1,1)};
    \addplot
      [
        domain = 0:1,
        samples = 100,
        color = blue,
        line width = 1pt,
        dashed,
      ]
    {1};
  \end{axis}
\end{tikzpicture}
\end{center}

\begin{center}
\begin{tikzpicture}
  \begin{axis}
    [
      axis lines = middle,
      xlabel = $x$,
      ylabel = $y$,
      xmax = 5,
      xmin = -5,
      ymax = 5,
      ymin = -5,
    ]
    \addplot
      [
        domain=-5:5,
        samples=100,
        color=blue,
      ] 
    {2};
    \addlegendentry{$f(x)=2$}
    \addplot
      [
        color = black,
        only marks,
        mark=*,
        mark size = 2pt,
        forget plot,
      ]
    coordinates {(1,2)};
    \addplot
      [
        domain = -5:5,
        samples = 100,
        color = blue,
        line width = 1pt,
        dashed,
      ]
      coordinates {(1,0)(1,2)};

  \end{axis}
\end{tikzpicture}
\end{center}

Básicamente, la suma de los valores de las ordenadas en las funciones $f(x)=x$ y $f(x)=2$ cuando $x \to 1$ es igual a una nueva coordenada que corresponde a la función $f(x)=x+2$, cuya ordenada es la solución al límite.

3) $\lim\limits_{x \to 2} \frac{x^2-4}{x-2} = \frac{2^2-4}{2-2} = \frac{0}{0}$, indeterminación. 

Quitemos la indeterminación: $\lim\limits_{x \to 2} \frac{x^2-4}{x-2} = \lim\limits_{x \to 2} \frac{(x-2)(x+2)}{x-2} = \lim\limits_{x \to 2} (x+2)=4$

\begin{center}
\begin{tikzpicture}
  \begin{axis}[
    axis lines = middle,
    xlabel = $x$,
    ylabel = $y$,
    ]
    \addplot[
      domain=-10:10,
      samples=100,
      color=blue,
      ] {x+2};
    \addlegendentry{$x+2$}
    \addplot
      [
        color = black,
        only marks,
        mark=o,
        mark size = 2pt,
        forget plot,
      ]
    coordinates {(2,4)};
  \end{axis}
\end{tikzpicture}
\end{center}

$\lim\limits_{x \to 2} \frac{x^2-4}{x-2} = x+2$, donde $x \neq 2$ y $Dom_f=\mathbb{R} \setminus \{2\}$

Para calcular la ordenada del "hoyo" (discontinuidad removible) en $x=2$, calculamos $\lim\limits_{x \to 2} f(x) = 4$. 

¿Cómo podemos redefinir la función para que sea continua en $x=2$? 

\textit{Def.} Continuidad en un punto: $f$ es continua en un punto $a \in Dom_f$ si $\lim\limits_{x \to a} f(x) = f(x)$.

Volviendo a nuestro caso anterior:

\[\text{Sea } g(x)= 
  \begin{cases}
    \frac{x^2-4}{x-2} & \text{si $x \neq 2$}\\ 
    4 & \text{si $x=2$}
  \end{cases}
\]

Esta es una nueva función que remueve la discontinuidad en $x=2$ para $\lim\limits_{x \to 2} \frac{x^2-4}{x-2}$, tratandose de una función por partes. 

\textbf{Exampli gratia:} Sea $f(x)=\frac{3x^3-7x^2+4x}{3x^3-10x^2+8x}$, estudie su continuidad. 

\[
f(x)=\frac{x(3x^2-7x+4)}{[x(3x^2-10x+8)}=\frac{x(3x-4)(x-1)}{x(3x-4)(x-2)}, Dom_f=\mathbb{R} \setminus \{0, \frac{4}{3}, 2\}
\]

\begin{center}
\begin{tikzpicture}
  \begin{axis}
    [
      axis lines = middle,
      xlabel = $x$,
      ylabel = $y$,
      xmax = 5,
      xmin = -5,
      ymax = 5,
      ymin = -5,
    ]
    \addplot
      [
        domain=-5:5,
        samples=100,
        color=blue,
      ] 
    {(1/(x-2))+1};
    \addlegendentry{$\frac{1}{x-2}=1$}
    \addplot
      [
       color=black,
       only marks,
       mark=o,
       mark size=2pt,
       forget plot,
      ]
    coordinates {(0,1/2)};
    \addplot
      [
       color=black,
       only marks,
       mark=o,
       mark size=2pt,
       forget plot,
      ]
    coordinates {(4/3,-1/2)};

  \addplot
    [
      domain = -5:5,
      samples = 100,
      color = blue,
      line width = 1pt,
      dashed,
    ]
    coordinates {(2,5)(2,-5)};
  \addplot
    [
      domain=-5:5,
      samples=100,
      color=blue,
      line width = 1pt,
      dashed,
    ] 
  {1};
  \end{axis}
\end{tikzpicture}
\end{center}

Evaluado en $x=0$,

\[
\lim\limits_{x \to 0} f(x) = \lim\limits_{x \to 0} \frac{x(3x-40(x-1)}{x(3x-4)(x-2)}
\]

\[
  \lim\limits_{x \to 0} \frac{x-1}{x-2}=\frac{1}{2} \text{, discontinuidad evitable en } (0,\frac{1}{2})
\]

Evaluado en $x=\frac{4}{3}$,

\[
\lim\limits_{x \to \frac{4}{3}} f(x)=\lim\limits_{x \to \frac{4}{3}} \frac{x(3x-4)(x-1)}{x(3x-4)(x-2)}
\]

\[
  \lim\limits_{x \to \frac{4}{3}} \frac{x-1}{x-2}=-\frac{1}{2} \text{, discontinuidad evitable en } (\frac{4}{3}, -\frac{1}{2})
\]

Evaluado en $x=2$,

\[
\lim\limits_{x \to 2} \frac{x-1}{x-2}=\frac{1}{0}=\infty \text{, está indefinido.}
\]

$\frac{1}{0}=\infty$ cuando se habla de limites, no en todos los contextos es verdad. 

$\frac{1}{0}=\infty \to \text{asíntota o discontinuidad infinita}$ 

Las discontinuidades infinitas no son evitables.

\[
  f(x)=\frac{x-1}{x-2} \text{, donde }x \neq 0, x \neq \frac{4}{3}; 
\]

\[
\frac{x-1}{x-2}=\frac{1}{x-2} + 1 
\]

Redefinamos la función para que sea continua en $x=0$ y $x=\frac{4}{3}$, 

\[\text{Sea } g(x)= 
  \begin{cases}
    f(x) & \text{si $x \neq 0, x\neq \frac{4}{3}$}\\ 
    \frac{x-1}{x-2} & \text{si $x=0 \lor x=4/3$}
  \end{cases}
\]

\[
Dom_g=\mathbb{R} \setminus \{2\} 
\]

\begin{center}
\begin{tikzpicture}
  \begin{axis}
    [
      axis lines = middle,
      xlabel = $x$,
      ylabel = $y$,
      xmax = 5,
      xmin = -5,
      ymax = 5,
      ymin = -5,
    ]
    \addplot
      [
        domain=-5:5,
        samples=100,
        color=blue,
      ] 
    {(1/(x-2))+1};
    \addlegendentry{$\frac{1}{x-2}=1$}
    \addplot
      [
       color=black,
       only marks,
       mark=*,
       mark size=2pt,
       forget plot,
      ]
    coordinates {(0,1/2)};
    \addplot
      [
       color=black,
       only marks,
       mark=*,
       mark size=2pt,
       forget plot,
      ]
    coordinates {(4/3,-1/2)};

  \addplot
    [
      domain = -5:5,
      samples = 100,
      color = blue,
      line width = 1pt,
      dashed,
    ]
    coordinates {(2,5)(2,-5)};
  \addplot
    [
      domain=-5:5,
      samples=100,
      color=blue,
      line width = 1pt,
      dashed,
    ] 
  {1};
  \end{axis}
\end{tikzpicture}
\end{center}

\textit{Nota:} Se refiere explícitamente a la continuidad en su propio dominio en una función, y la discontinuidad general considera a aquellos valores fuera del dominio. En este caso, la función \textit{es continua en su dominio}, mientras que \textit{es discontinua en $x=2$}.

\end{document}
