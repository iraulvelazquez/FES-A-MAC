\documentclass{article} % Document type 
\usepackage[spanish]{babel} % Language typesetting 
\usepackage[utf8]{inputenc} % Accented chars without commands
\usepackage[T1]{fontenc} % Accented display corrected
\selectlanguage{spanish} % Default lang, useful for for multi lang switching 
\usepackage{amsmath} % Math 1/3
\usepackage{amssymb} % Math 2/3
\usepackage{stmaryrd} % Math 3/3
\usepackage{tikz} % Crating graphics programmatically
\usepackage{pgfplots} % Extends tikz for 2D and 3D plots for data and math f 
\pgfplotsset{compat=1.18} % ---
\usepackage{circuitikz} % Extends tikz for electrical circuits
\usepackage[dvipsnames, svgnames]{xcolor} % Coloring
\usepackage{soul} % Text highlighting 
\setul{0.5ex}{0.3ex} % ---
\setlength{\parindent}{15pt} % Controls indentantion for the first line of paragraphs 
\setlength{\parskip}{0.8\baselineskip plus 0.2\baselineskip minus 0.1\baselineskip} % Vertical spacing between paragraphs 
\usepackage{geometry} % Layout
\geometry{margin=3cm} % --- 
\usepackage{titlesec} % Global section centering
\titleformat{\section} % --
  {\normalfont\Large\bfseries\centering}{\thesection}{1em}{} % --
\usepackage{changepage} % Special paragraph
\newenvironment{x}[3]{
  \begin{adjustwidth}{#1}{0cm}
  \setlength{\parindent}{0pt}
  #2\fontsize{#3}{1pt}\selectfont 
}{
  \end{adjustwidth}
}

\title{TVI, TVM y Teorema de Bolzano}
\date{21-11-2025}
\author{Velázquez Ramírez Carlos Raúl}

\begin{document}
\pagenumbering{gobble}
  \maketitle
  \newpage
  \pagenumbering{roman}

  \section*{Teorema del valor intermedio (TVI)} 

  Sea $f$ una función continua en $[a, b]$. Si $f(a) \neq f(b)$, entonces existe $c \in ]a, b[$ tal que $f(c)$ está entre $f(a)$ y $f(b)$. 

\begin{center}
\begin{tikzpicture}
  \begin{axis}
    [
      axis lines = middle,
      xlabel = $x$,
      ylabel = $f(x)$,
      xmax = 7,
      xmin = -1,
      ymax = 30,
      ymin = -1,
      xtick = \empty,
      ytick = \empty,
      extra x ticks={3, 4, 5},
      extra x tick labels={\small$a$, \small$c$, \small$b$},
      extra x tick style={tick label style={below, yshift = -2pt}},
      extra y ticks={9, 16, 25},
      extra y tick labels={\small$f(a)$, \small$f(c)$, \small$f(b)$},
      extra y tick style={tick label style={left, yshift = -2pt}},
    ]
    \addplot
      [
        domain=-1:9,
        samples=100,
        color=blue,
      ] 
    {x^2};
    \addplot
      [
       color=black,
       only marks,
       mark=*,
       mark size=2pt,
       forget plot,
      ]
    coordinates {(5,25)};
  \addplot
    [
      domain = -10:10,
      samples = 100,
      color = red,
      line width = 1pt,
      dashed,
    ]
    coordinates {(5,0)(5,25)};
  \addplot
    [
      domain=-0:5,
      samples=100,
      color=red,
      line width = 1pt,
      dashed,
    ] 
  {25};
    \addplot
      [
       color=black,
       only marks,
       mark=*,
       mark size=2pt,
       forget plot,
      ]
    coordinates {(4,16)};
  \addplot
    [
      domain = -10:10,
      samples = 100,
      color = red,
      line width = 1pt,
      dashed,
    ]
    coordinates {(4,0)(4,16)};
  \addplot
    [
      domain=-0:4,
      samples=100,
      color=red,
      line width = 1pt,
      dashed,
    ] 
  {16};
    \addplot
      [
       color=black,
       only marks,
       mark=*,
       mark size=2pt,
       forget plot,
      ]
    coordinates {(3,9)};
  \addplot
    [
      domain = -10:10,
      samples = 100,
      color = red,
      line width = 1pt,
      dashed,
    ]
    coordinates {(3,0)(3,9)};
  \addplot
    [
      domain=-0:3,
      samples=100,
      color=red,
      line width = 1pt,
      dashed,
    ] 
  {9};
  \end{axis}
\end{tikzpicture}
\end{center}

\textbf{Exampli gratia:} 

1) Considere la función: 

\[f(x)= 
  \begin{cases}
    x - 1 & \text{si $0 \leq x \leq 2$}\\ 
    x^2 & \text{si $2 < x \leq 3$}
  \end{cases}
\]

\begin{center}
\begin{tikzpicture}
  \begin{axis}
    [
      axis lines = middle,
      xlabel = $x$,
      ylabel = $y$,
      xmax = 4,
      xmin = -1,
      ymax = 7,
      ymin = -1,
    ]
    \addplot
      [
        domain=0:2,
        samples=100,
        color=blue,
      ] 
    {x-1};
    \addplot
      [
        domain = 2:3,
        samples = 100,
        color = blue,
      ]
    {x^2};
    \addplot
      [
        color = black,
        only marks,
        mark=*,
        mark size = 2pt,
        forget plot,
      ]
    coordinates {(2,1)};
    \addplot
      [
        color = black,
        only marks,
        mark=o,
        mark size = 2pt,
        forget plot,
      ]
    coordinates {(2,4)};
  \end{axis}
\end{tikzpicture}
\end{center}


\[
  f: [0, 3] \to \mathbb{R}
\]

\[
  f(0) = -1
\]

\[
  f(3) = 9
\]

\[
  f(2) = 1 \text{, $f$ no es continua en $x=2$,}
\]

\[
  2 \in ]0, 3[
\]

\[
  f(0) \leq f(2) \leq f(3) \Rightarrow -1 <= 1 <= 9
\]

2) Sea $g(x) = \frac{2}{x - 4}$, 

Nótese que $g$ es discontinua en \underline{$x = 4$}, el cual pertenece al intervalo \underline{$[2, 5]$}. 

Además, \underline{$g(2) = -1$} y \underline{$g(5) = 2$}. 

Si \underline{$k \in ]-1, 2[$}, no hay valor alguno de $c$ en \underline{$]2, 5[$}, tal que $g(c) = k$.

\begin{center}
\begin{tikzpicture}
  \begin{axis}
    [
      axis lines = middle,
      xlabel = $x$,
      ylabel = $y$,
      xmax = 7,
      xmin = -1,
      ymax = 5,
      ymin = -4,
      xtick = \empty,
      ytick = \empty,
      extra y ticks={-1, 1, 2},
      extra y tick labels={\small$g(2) = -1$, \small$g(6) = 1$, \small$g(5) = 2$},
      extra y tick style={tick label style={left, yshift = -2pt}},
    ]
    \addplot
      [
        domain=2:5,
        samples=50,
        color=blue,
      ] 
    {2/(x-4)};
  \addplot
    [
      domain = -10:10,
      samples = 100,
      color = red,
      line width = 1pt,
      dashed,
    ]
    coordinates {(2,-4)(2,5)};
  \addplot
    [
      domain = -10:10,
      samples = 100,
      color = red,
      line width = 1pt,
      dashed,
    ]
    coordinates {(5,-4)(5,5)};
  \end{axis}
\end{tikzpicture}
\end{center}


En particular, si $k = 1$, entonces $g(6) = 1$, pero $6 \notin [2, 5]$. 

\begin{center}
  \textit{Error}: $f$ es continua en $[2, 5]$ (\textit{falso}) $\Rightarrow 6 \notin ]2, 5[$
\end{center} 

3) Sea $f(x) = 4 + 3x - x^2$, $f: [2, 5] \to \mathbb{R}$. 

Verifique que el TVI se cumple para $f(c) = 1$ trazando la gráfica de $f$ y la recta de $y = 1$. 

\[
  f(2) = 6
\]

\[
  f(5) = -6
\]

Como $f(5) \leq f(c) \leq f(2)$, por el TVI, $\exists c \in ]2, 5[$ tal que $f(c) = 1$.

\[
  4 + 3x - x^2 = 1
\]

\[
  x^2 - 3x - 4 = -1
\]

\[
  x^2 - 3x - 3 = 0
\]

\[
  x_{1, 2} = \frac{3 \pm \sqrt{21}}{2}
\]

\[
  c_1 = \frac{2 + \sqrt{21}}{2}, \; c_2 = \frac{2 - \sqrt{21}}{2}
\]

\[
  f(x) = -x^2 + 3x + 4 = -(x^2 - 3x - 4) = -[x^2 - 3x + \frac{9}{4} - \frac{9}{4} - 4] = -[(x - \frac{3}{2})^2 - \frac{9}{4} - \frac{16}{4}]
\]

\[
  = -(x - \frac{3}{2})^2 + \frac{25}{4}
\]

\begin{center}
\begin{tikzpicture}
  \begin{axis}
    [
      axis lines = middle,
      xlabel = $x$,
      ylabel = $y$,
      xmax = 7,
      xmin = -4,
      ymax = 7,
      ymin = -1,
      xtick = \empty,
      ytick = \empty,
      extra x ticks={-0.8, 3.8},
      extra x tick labels={\small$c_2$, \small$c_1$},
      extra x tick style={tick label style={left, yshift = -2pt}},
    ]
    \addplot
      [
        domain=-7:7,
        samples=50,
        color=blue,
      ] 
    {-x^2+3*x+4};
    \addplot
      [
        domain=-7:7,
        samples=50,
        color=red,
      ] 
    {1};
  \addplot
    [
      domain = -10:10,
      samples = 100,
      color = red,
      line width = 1pt,
      dashed,
    ]
    coordinates {(-0.8,0)(-0.8,1)};
  \addplot
    [
      domain = -10:10,
      samples = 100,
      color = red,
      line width = 1pt,
      dashed,
    ]
    coordinates {(3.8,0)(3.8,1)};
    \addplot
      [
        color = black,
        only marks,
        mark=*,
        mark size = 2pt,
        forget plot,
      ]
    coordinates {(-0.8,1)};
    \addplot
      [
        color = black,
        only marks,
        mark=*,
        mark size = 2pt,
        forget plot,
      ]
    coordinates {(3.8,1)};
  \end{axis}
\end{tikzpicture}
\end{center}

Pero como $f:[2, 5]$, se deshecha $c_2$ pues no vive en el dominio, contrario a $c_1$.

\section*{Corolario: Teorema de Bolzano}

Sea $f$ continua en $[a, b]$ y sean $f(a)$ y $f(b)$ de signos opuestos. Entonces existe $c \in ]a,b[$ tal que $f(c) = 0$. 

\begin{center}
\begin{tikzpicture}
  \begin{axis}
    [
      axis lines = middle,
      xlabel = $x$,
      ylabel = $f(x)$,
      xmax = 6,
      xmin = -1,
      ymax = 10,
      ymin = -4,
      xtick = \empty,
      ytick = \empty,
      extra x ticks={1, 2.6, 3},
      extra x tick labels={\small$a$, \small$c$, \small$b$},
      extra x tick style={tick label style={below, yshift = -2pt}},
      extra y ticks={-2, 6},
      extra y tick labels={\small$f(b)$, \small$f(a)$},
      extra y tick style={tick label style={left, yshift = -2pt}},
    ]
    \addplot
      [
        domain=-1:9,
        samples=100,
        color=blue,
      ] 
    {-x^2+7};
    \addplot
      [
       color=black,
       only marks,
       mark=*,
       mark size=2pt,
       forget plot,
      ]
    coordinates {(3,-2)};
  \addplot
    [
      domain = -10:10,
      samples = 100,
      color = red,
      line width = 1pt,
      dashed,
    ]
    coordinates {(3,0)(3,-2)};
  \addplot
    [
      domain=0:3,
      samples=100,
      color=red,
      line width = 1pt,
      dashed,
    ] 
  {-2};
    \addplot
      [
       color=black,
       only marks,
       mark=*,
       mark size=2pt,
       forget plot,
      ]
    coordinates {(1,6)};
  \addplot
    [
      domain = -10:10,
      samples = 100,
      color = red,
      line width = 1pt,
      dashed,
    ]
    coordinates {(1,0)(1,6)};
  \addplot
    [
      domain=0:1,
      samples=100,
      color=red,
      line width = 1pt,
      dashed,
    ] 
  {6};
    \addplot
      [
       color=black,
       only marks,
       mark=*,
       mark size=2pt,
       forget plot,
      ]
    coordinates {(2.6,0)};
  \end{axis}
\end{tikzpicture}
\end{center}

Un ejemplo donde el Teorema de Bolzano no puede ser aplicado sería el siguiente: 

\begin{center}
\begin{tikzpicture}
  \begin{axis}
    [
      axis lines = middle,
      xlabel = $x$,
      ylabel = $f(x)$,
      xmax = 6,
      xmin = -1,
      ymax = 26,
      ymin = -1,
      xtick = \empty,
      ytick = \empty,
      extra x ticks={3, 4},
      extra x tick labels={\small$a$, \small$b$},
      extra x tick style={tick label style={below, yshift = -2pt}},
      extra y ticks={14, 21},
      extra y tick labels={\small$f(a)$, \small$f(b)$},
      extra y tick style={tick label style={left, yshift = -2pt}},
    ]
    \addplot
      [
        domain=-1:9,
        samples=100,
        color=blue,
      ] 
    {x^2 + 5};
    \addplot
      [
       color=black,
       only marks,
       mark=*,
       mark size=2pt,
       forget plot,
      ]
    coordinates {(4,21)};
  \addplot
    [
      domain = -10:10,
      samples = 100,
      color = red,
      line width = 1pt,
      dashed,
    ]
    coordinates {(4,0)(4,21)};
  \addplot
    [
      domain=-0:4,
      samples=100,
      color=red,
      line width = 1pt,
      dashed,
    ] 
  {21};
    \addplot
      [
       color=black,
       only marks,
       mark=*,
       mark size=2pt,
       forget plot,
      ]
    coordinates {(3,14)};
  \addplot
    [
      domain = -10:10,
      samples = 100,
      color = red,
      line width = 1pt,
      dashed,
    ]
    coordinates {(3,0)(3,14)};
  \addplot
    [
      domain=-0:3,
      samples=100,
      color=red,
      line width = 1pt,
      dashed,
    ] 
  {14};
  \end{axis}
\end{tikzpicture}
\end{center}

Donde la gráfica no cruza el eje $x$.

\textbf{Exampli gratia:} 

1) Estudie las raíces del polinomio $p(x) = x^3 - x^2 + 1$.  

Por medio del Teorema de Descartes: 

\begin{center}
$p(x) \Rightarrow$ $2$ o $0$ raíces positivas $\Rightarrow$ $0$ raíces complejas.

$p(-x) \Rightarrow$ $1$ raíz negativa $\Rightarrow$ $2$ raíces complejas.
\end{center}

\[
  p(-1) = -1
\]

\[
  p(1) = 1
\]

\[
  p(0) = 1
\]

La raíz negativa, por el teorema de Bolzano, está en $]-1, 0[$.

Por medio del método de bisección: 

\[
  p(- \frac{1}{2}) = (- \frac{1}{2})^3 - (- \frac{1}{2})^2 + 1 = - \frac{1}{8} - \frac{1}{4} + 1 = \frac{5}{8} > 0
\]

\section*{Teorema del valor medio (TVM)}

Sea $f$ continua en $[a, b]$ y derivable en $]a, b[$. Entonces, existe un punto $c \in ]a, b[$ tal que 

\[
  f'(c) = \frac{f(b) - f(a)}{b - a}.
\]

Donde $f'(c)$ es la pendiente de la recta tangente, y $\frac{f(b) - f(a)}{b - a}$ es la pendiente del segmento de recta que une $(a, f(a))$ con $(b, f(b))$. 


\begin{center}
\begin{tikzpicture}
  \begin{axis}
    [
      axis lines = middle,
      xlabel = $x$,
      ylabel = $f(x)$,
      xmax = 6,
      xmin = -1,
      ymax = 8,
      ymin = -1,
      xtick = \empty,
      ytick = \empty,
      extra x ticks={0.6, 2, 3.4},
      extra x tick labels={\small$a$, \small$c$, \small$b$},
      extra x tick style={tick label style={below, yshift = -2pt}},
      extra y ticks={0.2, 5.8},
      extra y tick labels={\small$f(a)$, \small$f(b)$},
      extra y tick style={tick label style={left, yshift = -2pt}},
    ]
    \addplot
      [
        domain=-1:9,
        samples=100,
        color=blue,
        line width = 2pt,
      ] 
    {-(x-3)^2+6};
    \addplot
      [
        domain=-1:9,
        samples=100,
        color=green,
      ] 
    {2*x+1};
    \addplot
      [
        domain=-1:9,
        samples=100,
        color=blue,
      ] 
    {2*x-1};
    \addplot
      [
       color=black,
       only marks,
       mark=*,
       mark size=2pt,
       forget plot,
      ]
    coordinates {(3.4,5.8)};
  \addplot
    [
      domain = -10:10,
      samples = 100,
      color = red,
      line width = 1pt,
      dashed,
    ]
    coordinates {(3.4,0)(3.4,5.8)};
  \addplot
    [
      domain=0:3.4,
      samples=100,
      color=red,
      line width = 1pt,
      dashed,
    ] 
  {5.8};
    \addplot
      [
       color=black,
       only marks,
       mark=*,
       mark size=2pt,
       forget plot,
      ]
    coordinates {(0.6,0.2)};
  \addplot
    [
      domain = -10:10,
      samples = 100,
      color = red,
      line width = 1pt,
      dashed,
    ]
    coordinates {(0.6,0)(0.6,0.2)};
  \addplot
    [
      domain=0:0.6,
      samples=100,
      color=red,
      line width = 1pt,
      dashed,
    ] 
  {0.2};
    \addplot
      [
       color=black,
       only marks,
       mark=*,
       mark size=2pt,
       forget plot,
      ]
    coordinates {(2,5)};
  \addplot
    [
      domain = -10:10,
      samples = 100,
      color = red,
      line width = 1pt,
      dashed,
    ]
    coordinates {(2,0)(2,5)};
  \end{axis}
\end{tikzpicture}
\end{center}

\textbf{Exampli gratia:} Determine los valores $a, b, c \in \mathbb{R}$ tal que $f$ satisfaga la hipótesis del TVM en $[0, 3]$.

\[f(x)= 
  \begin{cases}
    1 & \text{, si } x = 0\\ 
    ax + b & \text{, si } 0 < x \leq 1\\ 
    x^2 + 4x + c & \text{, si } 1 < x \leq 3
  \end{cases}
\]

Primero, estudiemos la continuidad de $f$ en $[0, 3]$. Como $f(x) = ax + b$ en $]0, 1[$, $f$ es continua por se un polinomio. 

Como $f(x) = x^2 + 4x + c$ en $]1, 3[$, $f$ es continua por ser un polinomio. 

Notemos que $f$ es continua por izq. en $x = 3$. 

Vemos las condiciones para que $f$ sea continua en $x = 0$ y continua en $x = 1$. 

\[
  f(0) = 1, \; f(1) = a + b
\]

\[
  \lim\limits_{x \to 0^+} f(x) = \lim\limits_{x \to 0, \, x>0} (ax + b) = b; \; f(0) = \lim\limits_{x \to 0^+} f(x) \Leftrightarrow b = 1
\]

\[
  \lim\limits_{x \to 1^-} f(x) = \lim\limits_{x \to 1, \, x<1} (ax + b) = a + b
\]

\[
  \lim\limits_{x \to 1^+} f(x) = \lim\limits_{x \to 1, \, x>1} (x^2 + 4x + c) = 5 + c
\]

De $f(1) = \lim\limits_{x \to 1} f(x)$, se tiene que $f(1) = \lim\limits_{x \to 1^-} f(x) = \lim\limits_{x \to 1^+} f(x)$.

\[
  a + b = a + b = 5 + c \Leftrightarrow a + b = 5 + c
\]

Pero $b = 1 \Rightarrow a + 1 = 5 + c \Rightarrow a = c + 4$. 

Ahora, estudiemos la derivabilidad de $f$ en $]0, 5[$, ¿qué pasa en $x = 1$? 

Calculemos $f'(1)$ por medio de sus derivadas laterales: 

\[
  f'(1^-) = \lim\limits_{x \to 1^-} \frac{f(x) - f(1)}{x - 1} = \lim\limits_{x \to 1, \, x<1} \frac{(ax + b) - (a + b)}{x - 1} = \lim\limits_{x \to 1, \, x<1} \frac{a(x - 1)}{x - 1} = a
\]

\[
  f'(1^+) \lim\limits_{x to 1^+} \frac{f(x) = f(1)}{x - 1} = \lim\limits_{x \to 1, \, x>1} \frac{(x^2 + 4x + c) - (a + b)}{x - 1}
\]

\[
  \text{Considerando $b = 1$ } \Rightarrow \lim\limits_{x \to 1, \, x>1} \frac{x^2 + 4x + c - a - 1}{x - 1}
\]

\[
  \text{Considerando $c - a = -4$ } \Rightarrow \lim\limits_{x to 1, \, x>1} \frac{x^2 + 4x + (-4) - 1}{x - 1} = \lim\limits_{x \to 1, \, x>1} \frac{x^2 + 4x - 5}{x - 1} 
\]

\[
  = \lim\limits_{x \to 1, \, x>1} \frac{(x + 5)(x - 1)}{x - 1} = 6
\]

\end{document}
