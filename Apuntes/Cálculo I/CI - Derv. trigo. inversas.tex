\documentclass{article} % Document type 
\usepackage[spanish]{babel} % Language typesetting 
\usepackage[utf8]{inputenc} % Accented chars without commands
\usepackage[T1]{fontenc} % Accented display corrected
\selectlanguage{spanish} % Default lang, useful for for multi lang switching 
\usepackage{amsmath} % Math 1/3
\usepackage{amssymb} % Math 2/3
\usepackage{stmaryrd} % Math 3/3
\usepackage{tikz} % Crating graphics programmatically
\usepackage{pgfplots} % Extends tikz for 2D and 3D plots for data and math f 
\pgfplotsset{compat=1.18} % ---
\usepackage{circuitikz} % Extends tikz for electrical circuits
\usepackage[dvipsnames, svgnames]{xcolor} % Coloring
\usepackage{soul} % Text highlighting 
\setul{0.5ex}{0.3ex} % ---
\setlength{\parindent}{15pt} % Controls indentantion for the first line of paragraphs 
\setlength{\parskip}{0.8\baselineskip plus 0.2\baselineskip minus 0.1\baselineskip} % Vertical spacing between paragraphs 
\usepackage{geometry} % Layout
\geometry{margin=3cm} % --- 
\usepackage{titlesec} % Global section centering
\titleformat{\section} % --
  {\normalfont\Large\bfseries\centering}{\thesection}{1em}{} % --
\usepackage{changepage} % Special paragraph
\newenvironment{x}[3]{
  \begin{adjustwidth}{#1}{0cm}
  \setlength{\parindent}{0pt}
  #2\fontsize{#3}{1pt}\selectfont 
}{
  \end{adjustwidth}
}

\DeclareMathOperator{\arccot}{arccot}

\title{Diferenciación de funciones trigonométricas inversas}
\date{19-11-2025}
\author{Velázquez Ramírez Carlos Raúl}

\begin{document}
\pagenumbering{gobble}
  \maketitle
  \newpage
  \pagenumbering{roman}

\section*{Teorema de la función inversa}

\textit{Def.} Sea $y = f(x)$ una función biyectiva. Suponemos que existe $f^{-1} (x)$ y es derivable, entonces 

\[
  \frac{dy}{dx} = \frac{1}{\frac{dx}{dy}}
\]

\textbf{Exampli gratia:}

1) Sea $y = \arcsin(x)$, ¿cómo obtener $y'$? 

Por el teorema de la función inversa, 

\[
  y' = \frac{dy}{dx} = \frac{1}{\frac{dx}{dy}} 
\]

\[
  \arcsin(x) \Rightarrow \sin(y) = \sin(\arcsin(x)) \Rightarrow \sin(y) = x
\]

\[
  x = \sin(y) \Rightarrow \frac{dx}{dy} = \frac{d \sin(y)}{dy} = \cos(y)
\]

\[
  \frac{dx}{dy} = \cos(y)
\]

\[
  y' = \frac{1}{\cos(y)}
\]

Consideremos que, 

\[
  \sin^2(x) + \cos^2(x) = 1
\]

\[
  \cos^2(x) = 1 - \sin^2(x) 
\]

\[
  \cos(x) = \sqrt{1 - \sin^2(x)}
\]

Por lo tanto, 

\[
  y' = \frac{1}{\sqrt{1 - \sin^2(y)}} = \frac{1}{\sqrt{1 - x^2}}
\]

\[
  \therefore \frac{d (\arcsin(x))}{dx} = \frac{1}{\sqrt{1 - x^2}}
\]

Si se combina el teorema de la función inversa con la regla de la cadena se tiene que: 

\[
  \frac{d}{dx} \arcsin(v) = \frac{v'}{\sqrt{1 - v^2}}
\]

, donde $v = v(x)$.

Ejercicio: Obtener $\frac{d}{dx} \arccos(x)$, 

2)

\[
  y = \arctan(x) \Rightarrow x = \cot(y) \Rightarrow x' = -\csc^2y
\]

\[
  \frac{dy}{dx} = \frac{1}{\frac{dx}{dy}} = \frac{1}{- \csc^2(y)}
\]

Debemos pasarla a terminos de $x$.

Consideremos $\sin^2(x) + \cos^2(x) = 1 \Rightarrow 1 + \cot^2(x) = \csc^2(x)$, al dividir entre $\sin^2(x)$.

Entonces, 

\[
  = \frac{-1}{1 + \cot^2(y)} = \frac{-1}{1 + x^2}
\]

\[
  \therefore \frac{d}{dx} (\arccot(x)) = \frac{-1}{1 + x^2}
\]

\end{document}
