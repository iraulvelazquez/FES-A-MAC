\documentclass{article} % Document type 
\usepackage[spanish]{babel} % Language typesetting 
\usepackage[utf8]{inputenc} % Accented chars without commands
\usepackage[T1]{fontenc} % Accented display corrected
\selectlanguage{spanish} % Default lang, useful for for multi lang switching 
\usepackage{amsmath} % Math 1/3
\usepackage{amssymb} % Math 2/3
\usepackage{stmaryrd} % Math 3/3
\usepackage{tikz} % Crating graphics programmatically
\usepackage{pgfplots} % Extends tikz for 2D and 3D plots for data and math f 
\pgfplotsset{compat=1.18} % ---
\usepackage{circuitikz} % Extends tikz for electrical circuits
\usepackage[dvipsnames, svgnames]{xcolor} % Coloring
\usepackage{soul} % Text highlighting 
\setul{0.5ex}{0.3ex} % ---
\setlength{\parindent}{15pt} % Controls indentantion for the first line of paragraphs 
\setlength{\parskip}{0.8\baselineskip plus 0.2\baselineskip minus 0.1\baselineskip} % Vertical spacing between paragraphs 
\usepackage{geometry} % Layout
\geometry{margin=3cm} % --- 
\usepackage{titlesec} % Global section centering
\titleformat{\section} % --
  {\normalfont\Large\bfseries\centering}{\thesection}{1em}{} % --
\usepackage{changepage} % Special paragraph
\newenvironment{x}[3]{
  \begin{adjustwidth}{#1}{0cm}
  \setlength{\parindent}{0pt}
  #2\fontsize{#3}{1pt}\selectfont 
}{
  \end{adjustwidth}
}

\title{Límites y definicón formal}
\date{27-10-2025}
\author{Velázquez Ramírez Carlos Raúl}

\begin{document}
\pagenumbering{gobble}
  \maketitle
  \newpage
  \pagenumbering{roman}

\section*{Logaritmo natural y exponencial natural}

\[
  \ln : ]0, \infty[ \to \mathbb{R}
\]

\[
  x \mapsto \ln x
\]

\[
  Img_{ln} = \mathbb{R}
\]

\begin{center}
\begin{tikzpicture}
  \begin{axis}
    [
      axis lines = middle,
      xlabel = $x$,
      ylabel = $y$,
      xmax = 5,
      xmin = -1,
      ymax = 2,
      ymin = -3,
    ]
    \addplot
      [
        domain=-2:5,
        samples=100,
        color=blue,
      ] 
    {ln(x)};
  \addlegendentry{$\ln x$}
  \end{axis}
\end{tikzpicture}
\end{center}

\textit{Propiedades del logaritmo natural:}

\[
  \ln a^x = x \ln a
\]

\[
  \ln ab = \ln a + \ln b
\]

\[
  \ln \frac{a}{b} = \ln a - \ln b
\]

\[
  \ln e = 1
\]

\[
  \ln (1) = 0
\]

\textit{Función exponencial natural:}

\[
  exp : \mathbb{R} \to \mathbb{R}
\]

\[
  x \mapsto e^x 
\]

\[
  Im_{ln} = ]0, \infty[
\]

\begin{center}
\begin{tikzpicture}
  \begin{axis}
    [
      axis lines = middle,
      xlabel = $x$,
      ylabel = $y$,
      xmax = 3,
      xmin = -5,
      ymax = 5,
      ymin = -1,
    ]
    \addplot
      [
        domain=-5:3,
        samples=100,
        color=blue,
      ] 
    {e^x};
  \addlegendentry{$e^x$}
  \end{axis}
\end{tikzpicture}
\end{center}

Si $exp$ se define como $exp : \mathbb{R} \to ]0, \infty[$, entonces $exp$ y $\ln$ son una inversa de la otra. 

\[
  \ln(e^x) = x
\]

\[
  \lim\limits_{x \to 0^+} \ln x = -\infty
\]

\[
  \lim\limits_{x \to 0^-} \ln x \text{ no puede evaluarse} 
\]

\[
  \lim\limits_{x \to 0} \ln x \text{, esta expresión no tiene sentido} 
\]

\[
  \lim\limits_{x \to -\infty} e^x = 0
\]

\[
  \lim\limits_{x \to \infty} e^x = \infty
\]

\[
  \lim\limits_{x \to 0} e^x = 1
\]

\[
  \lim\limits_{x \to -\infty} \frac{1}{e^{-x}} = \lim\limits_{x \to \infty} e^x = 0
\]

\[
  \lim\limits_{x \to -\infty} e^{-x} = \infty 
\]

\[
  \lim\limits_{x \to \infty} e^{-x} = 0
\]

\begin{center}
\begin{tikzpicture}
  \begin{axis}
    [
      axis lines = middle,
      xlabel = $x$,
      ylabel = $y$,
      xmax = 5,
      xmin = -5,
      ymax = 5,
      ymin = -5,
    ]
    \addplot
      [
        domain=-5:5,
        samples=100,
        color=blue,
      ] 
    {e^x};
  \addlegendentry{$e^x$}
    \addplot
      [
        domain=-5:5,
        samples=100,
        color=red,
      ] 
    {ln(x)};
  \addlegendentry{$\ln x$}
    \addplot 
      [
        domain=-5:5,
        samples=100,
        color=purple,
      ]
    {x};
  \end{axis}
\end{tikzpicture}
\end{center}

\section*{Otras funciones trigonométricas}

\[
  f(x) = \tan(x)
\]

\[
  Dom_{\tan(x)} = \mathbb{R} \setminus \{ \frac{(2k + 1) \pi}{2} \, | \, k \in \mathbb{Z}\}
\]

\[
  Img_{\tan(x)} = \mathbb{R}
\]

\begin{center}
\begin{tikzpicture}
  \begin{axis}
    [
      axis lines = middle,
      xlabel = $x$,
      ylabel = $y$,
      xmax = 5,
      xmin = -5,
      ymax = 5,
      ymin = -5,
    ]
    \addplot
      [
        domain=-1.5*pi:1.5*pi,
        samples=150,
        color=blue,
      ] 
    {tan(deg(x))};
  \addlegendentry{$\tan(x)$}
  \end{axis}
\end{tikzpicture}
\end{center}

\[
  f(x) = \arctan(x)
\]

\[
  Dom_g = \mathbb{R}
\]

\[
  Img_g = ]-\frac{\pi}{2}, \frac{\pi}{2}[
\]
\begin{center}
\begin{tikzpicture}
  \begin{axis}
    [
      axis lines = middle,
      xlabel = $x$,
      ylabel = $y$,
      xmax = 5,
      xmin = -5,
      ymax = 3,
      ymin = -3,
    ]
    \addplot
      [
        domain=-5:5,
        samples=250,
        color=blue,
      ] 
    {rad(atan(x))};
  \addlegendentry{$\arctan(x)$}
  \end{axis}
\end{tikzpicture}
\end{center}

Si restringimos la función $f(x) = \tan x$ al intervalo $]-\frac{\pi}{2}, \frac{\pi}{2}[$, entonces $\tan x$ y $\arctan x$ son funciones inversas una de la otra. 

\[
  \tan(\arctan x) = x
\]

\[
  \arctan(\tan x) = x
\]

\section*{Límites (definicón rigurosa)} 

\textit{Def. Vecindad, entorno o recinto:} Sea $a \in \mathbb{R}$ y $\epsilon > 0$. La vecindad del punto $a$ con radio $\epsilon$, $V_{\epsilon} (a)$, se define como 

\[
  V_{\epsilon} (a) := \{x \in \mathbb{R} \, | \, |x - a| < \epsilon\}
\]

\textbf{Exampli gratia:}

1) $V_{1} (\frac{3}{4}) := \{x \in \mathbb{R} \, | \, |x - \frac{3}{4}| < \epsilon\}$

\[
  |x - \frac{3}{4}| < \epsilon \text{, por propiedades del valor absoluto, se tiene que}
\]

\[
  -\epsilon < x - \frac{3}{4} < \epsilon
\]

\[
  \frac{3}{4} - \epsilon < x < \frac{3}{4} + \epsilon
\]

\[
  x \in ]\frac{3}{4} - \epsilon, \frac{3}{4} + \epsilon[
\]

2) Exprese el intervalo $]2, 6[$ como una vecindad. 

\[
  ]2, 6[ = V_2 (4)
\]

\textit{Def. Existencia del límite:} $\lim\limits_{x \to a} f(x) = L$ significa que $\forall \epsilon > 0 \; \exists \delta > 0 \, | \, \forall  x \in Dom_f \, | \, 0 < |x - a| < \delta \Rightarrow |f(x) - L| < \epsilon$.

Lo cual quiere decir que $x \in V_{\delta} (a) \Rightarrow f(x) \in V_{\epsilon} (L)$.

\begin{center}
\begin{tikzpicture}
  \begin{axis}
    [
      axis lines = left,
      xlabel = $x$,
      ylabel = $y$,
      xmax = 9,
      xmin = 0,
      ymax = 30,
      ymin = 0,
      extra x ticks={5},
      extra x tick labels={$a$},
      extra x tick style={tick label style={below}},
      extra y ticks={25},
      extra y tick labels={$L$},
      extra y tick style={tick label style={left}},
    ]
    \addplot
      [
        domain=0:9,
        samples=100,
        color=blue,
      ] 
    {x^2};
    \addlegendentry{$x^2$}
    \addplot
      [
       color=black,
       only marks,
       mark=*,
       mark size=2pt,
       forget plot,
      ]
    coordinates {(5,25)};
  \addplot
    [
      domain = -10:10,
      samples = 100,
      color = blue,
      line width = 1pt,
      dashed,
    ]
    coordinates {(5,0)(5,25)};
  \addplot
    [
      domain=-0:5,
      samples=100,
      color=blue,
      line width = 1pt,
      dashed,
    ] 
  {25};
  \end{axis}
\end{tikzpicture}
\end{center}

\textbf{Exampli gratia:} 

1) Demuestre que $\lim\limits_{x \to 2} (x+3) = 5$. 

Demostración: Sea $\epsilon > 0$ arbitrario. Para encontrar la $\delta > 0$, estimamos 

\[
  |f(x) - L| = |x + 3 - 5| = |x - 2| < \epsilon
\]

Si $|x - 2| < \delta$, entonces $|f(x) - L| < \delta = \epsilon$. 

2) Demuestre que $\lim\limits_{x \to 4} (\frac{1}{2} x + 1) = 3$.

Demostración: Sea $\epsilon > 0$ arbitrario. Si $\delta = 2 \epsilon$. Supongamos que $|x - 4| < \delta$. Se verifica $|f(x) - L| = |\frac{1}{2} x +1 - 3| = |\frac{1}{2} x - 2| = |\frac{1}{2} (x - 4)| = \frac{1}{2} |x - 4| < \frac{1}{2} \delta = \epsilon$. 
\end{document}
