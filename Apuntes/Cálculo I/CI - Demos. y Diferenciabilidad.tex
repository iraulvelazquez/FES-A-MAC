\documentclass{article} % Document type 
\usepackage[spanish]{babel} % Language typesetting 
\usepackage[utf8]{inputenc} % Accented chars without commands
\usepackage[T1]{fontenc} % Accented display corrected
\selectlanguage{spanish} % Default lang, useful for for multi lang switching 
\usepackage{amsmath} % Math 1/3
\usepackage{amssymb} % Math 2/3
\usepackage{stmaryrd} % Math 3/3
\usepackage{tikz} % Crating graphics programmatically
\usepackage{pgfplots} % Extends tikz for 2D and 3D plots for data and math f 
\pgfplotsset{compat=1.18} % ---
\usepackage{circuitikz} % Extends tikz for electrical circuits
\usepackage[dvipsnames, svgnames]{xcolor} % Coloring
\usepackage{soul} % Text highlighting 
\setul{0.5ex}{0.3ex} % ---
\setlength{\parindent}{15pt} % Controls indentantion for the first line of paragraphs 
\setlength{\parskip}{0.8\baselineskip plus 0.2\baselineskip minus 0.1\baselineskip} % Vertical spacing between paragraphs 
\usepackage{geometry} % Layout
\geometry{margin=3cm} % --- 
\usepackage{titlesec} % Global section centering
\titleformat{\section} % --
  {\normalfont\Large\bfseries\centering}{\thesection}{1em}{} % --
\usepackage{changepage} % Special paragraph
\newenvironment{x}[3]{
  \begin{adjustwidth}{#1}{0cm}
  \setlength{\parindent}{0pt}
  #2\fontsize{#3}{1pt}\selectfont 
}{
  \end{adjustwidth}
}

\title{Derivación, Demostración, y Diferenciabilidad}
\date{12-11-2025}
\author{Velázquez Ramírez Carlos Raúl}

\begin{document}
\pagenumbering{gobble}
  \maketitle
  \newpage
  \pagenumbering{roman}

\section*{Derivación de funciones trigonométricas} 

Tenemos que 

\noindent 1) $\sin(\alpha+ \beta) = \sin(\alpha) \cos(\beta) + \sin(\beta) \cos(\alpha)$.\\ 
2) $\sin(\alpha - \beta) = \sin(\alpha) \cos(\beta) - \sin(\beta) \cos(\alpha)$.

De $1 + 2$ obtenemos que $\sin(\alpha + \beta) + \sin(\alpha - \beta) = 2 \sin(\alpha) \cos(\beta)$.

Sean $A = \alpha + \beta$ y $B = \alpha - \beta$, 

\[
  \sin(A) + \sin(B) = 2 \sin(\frac{A + B}{2}) \cos(\frac{A - B}{2})
\]

De $1 - 2$ obtenemos que $\sin(\alpha + \beta) - \sin(\alpha - \beta) = 2 \sin(\alpha) \cos(\beta)$.

\[
  \sin(A) - \sin(B) = 2 \sin(\frac{A - B}{2}) \cos(\frac{A + B}{2})
\]

Demuestre que $f(x) = \sin(x)$ es diferenciable en $\mathbb{R}$ y calcule su derivada. 

Demostración: Sean $x \neq x_0$ y $x, x_0 \in \mathbb{R}$, 

\[
  \frac{f(x) - f(x_0)}{x - x_0} = \frac{\sin(x) - \sin(x_0)}{x - x_0} = \frac{2 \sin(\frac{x - x_0}{2}) \cos(\frac{x + x_0}{2})}{x - x_0} = \frac{\sin(\frac{x - x_0}{2})}{\frac{x - x_0}{2}} \cdot \cos(\frac{x + x_0}{2})
\]

Así, 

\[
  \lim\limits_{x \to x_0} \frac{f(x) - f(x_0)}{x - x_0} = \lim\limits_{x \to x_0} \frac{\sin(\frac{x - x_0}{2})}{\frac{x - x_0}{2}} \cdot \cos(\frac{x + x_0}{2}) = (1) \cdot \cos(\frac{x + x_0}{2}) = \cos(x_0) \Rightarrow f'(x) = \cos(x_0)
\]

Así, $f$ es diferenciable en $\mathbb{R}$ y $f'(x) = \cos(x)$.

\section*{Derivación de $n$ funciones} 

\textit{Proposición:} Sean $f, g: ] \alpha, \beta [ \to \mathbb{R}$ diferenciable en $x_0 \in ] \alpha, \beta [$. Entonces, 

\noindent 1) $f + g$ es diferenciable en $x_0$ y $(f + g)'(x_0) = f'(x_0) + g'(x_0)$.\\
2) $(fg)'$ es diferenciable y $(fg)'(x_0) = f'(x_0) g(x_0) + f(x_0) g'(x_0)$. \\
3) Si $g(x_0) \neq 0$, entonces $\exists ] \alpha, \beta [ \subset ] \alpha, \beta [$ con $x_0 \in ] \alpha, \beta [$.

La función $\frac{1}{g}$, definida, al menos en $] \alpha, \beta[$, es diferenciable en $x_0$ y, además, 

\[
  (\frac{1}{g}')(x_0) = \frac{-g'(x_0)}{g^2(x_0)}
\]

\noindent 4) Si $g(x_0) \neq 0$, $\frac{f}{g}$ es diferenciable en $x_0$ y además, 

\[
  (\frac{f}{g})'(x_0) = \frac{g(x_0) f'(x_0) - f(x_0) g'(x_0)}{g^2(x_0)}
\]

Demostración: 

1) Tenemos que para $x \neq x_0$, 

\[
  \frac{(f + g)(x) - (f + g)(x_0)}{x - x_0} = \frac{(f(x) - g(x)) + (f(x_0) - g(x_0))}{x - x_0} = \frac{f(x) - f(x_0)}{x - x_0} + \frac{g(x) - g(x_0)}{x - x_0} 
\]
Entonces, 

\[
  \lim\limits_{x \to x_0} \frac{(f + g)(x) - (f + g)(x_0)}{x - x_0} = \lim\limits_{x \to x_0} \frac{f(x) - f(x_0)}{x - x_0} = f'(x_0) + g'(x_0)
\]

2) Sea $x \neq x_0$,

\[
  \lim\limits_{x \to x_0} \frac{(fg)(x) - (fg)(x_0)}{x - x_0} = \lim\limits_{x \to x_0} g(x) \cdot \lim\limits_{x \to x_0} \frac{f(x) - f(x_0)}{x - x_0} + \lim\limits_{x \to x_0} f(x_0) \cdot \lim\limits_{x \to x_0} \frac{g(x) - g(x_0)}{x - x_0}
\]

\[
  = g(x_0) \cdot f'(x_0) + f(x_0) \cdot g'(x_0)
\]

\section*{Diferenciabilidad} 

\textit{Teorema:} La continuidad es una condición necesaria para la diferenciabilidad.

\[
  p \to q
\]

\begin{center}
  $p$ es condición suficiente para $q$ 

  $q$ es condición necesaria para $p$

  $p$: $f$ es diferenciable. 

  $q$: $f$ es continua.

Que $f$ sea continua es condición necesaria más no suficiente para ser diferenciable.
\end{center}

\textbf{Exampli gratia:} En el caso $f(x) = |x|$, $f'(0)$ no existe.

\[
  \lim\limits){x \to 0} \frac{f(x) - f(0)}{x - 0}
\]

Con límites laterales: 

\[
  \lim\limits_{x \to 0^-} \frac{-x}{x} = -1
\]

\[
  \lim\limits_{x \to 0^+} \frac{x}{x} = 1
\]

\[
  \lim\limits_{x \to 0^+} f(x) \neq \lim\limits_{x \to 0^-} f(x)
\]

$f$ es continua en $\mathbb{R}$ pero no es derivable en $x = 0$. 

Sea $f: ] \alpha, \beta [ \to \mathbb{R}$ diferenciable en $x_0 \in ] \alpha, \beta [$, entonces $f$ es continua en $x_0$.

Demostración: 

Suponemos que $\lim\limits_{x \to x_0} \frac{f(x) - f(x_0)}{x - x_0}$ existe. 

Debemos mostrar qie $\lim\limits_{x \to x_0} f(x) = f(x_0)$.

Para $x \neq x_0$, 

\[
f(x) - f(x_0) = (f(x) - f(x_0)) \frac{x - x_0}{x - x_0} = \frac{f(x) - f(x_0)}{x - x_0} (x - x_0)
\]

Entonces

\[
  \lim\limits_{x \to x_0} [f(x) - f(x_0)] = \lim\limits_{x \to x_0} \frac{f(x) - f(x_0)}{x - x_0} \cdot \lim\limits_{x \to x_0} (x - x_0) = f'(x_0) \cdot 0 = 0
\]

\[
  \Rightarrow \lim\limits_{x \to x_0} [f(x) - f(x_0)] = \lim\limits_{x \to x_0} f(x) - \lim\limits_{x \to x_0} = 0
\]

\[
  \Rightarrow \lim\limits_{x \to x_0} f(x) = f(x_0) \textit{, Q.E.D}
\]

\textit{Obs.:} 

\[
  (f_1 f_2)' = f_1 f_2' + f_1' f_2
\]

\[
  (f_1 f_2 f_3)' = (f_1 (f_2 f_3))' = f_1 (f_2 f_3)' + f_1' f_2 f_3 = f_1 (f_2 f_3' + f_2' f_3) + f_1' f_2 f_3 = f_1 f_2 f_3' + f_1 f_2' f_3 + f_1' f_2 f_3 
\]

En general: 

\[
  (f_1 f_2 \dots f_n)' = \sum_{i = 1}^{n} f_1 f_2 \dots f_{i -1} f_i' f_{i + 1} \dots f_n
\]


Demostrar por inducción sobre $n \in \mathbb{N}$ y después programarla.

\end{document}

