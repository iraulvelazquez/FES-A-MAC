\documentclass{article} % Document type 
\usepackage[spanish]{babel} % Language typesetting 
\usepackage[utf8]{inputenc} % Accented chars without commands
\usepackage[T1]{fontenc} % Accented display corrected
\selectlanguage{spanish} % Default lang, useful for for multi lang switching 
\usepackage{amsmath} % Math 1/3
\usepackage{amssymb} % Math 2/3
\usepackage{stmaryrd} % Math 3/3
\usepackage{tikz} % Crating graphics programmatically
\usepackage{pgfplots} % Extends tikz for 2D and 3D plots for data and math f 
\pgfplotsset{compat=1.18} % ---
\usepackage{circuitikz} % Extends tikz for electrical circuits
\usepackage[dvipsnames, svgnames]{xcolor} % Coloring
\usepackage{soul} % Text highlighting 
\setul{0.5ex}{0.3ex} % ---
\setlength{\parindent}{15pt} % Controls indentantion for the first line of paragraphs 
\setlength{\parskip}{0.8\baselineskip plus 0.2\baselineskip minus 0.1\baselineskip} % Vertical spacing between paragraphs 
\usepackage{geometry} % Layout
\geometry{margin=3cm} % --- 
\usepackage{titlesec} % Global section centering
\titleformat{\section} % --
  {\normalfont\Large\bfseries\centering}{\thesection}{1em}{} % --
\usepackage{changepage} % Special paragraph
\newenvironment{x}[3]{
  \begin{adjustwidth}{#1}{0cm}
  \setlength{\parindent}{0pt}
  #2\fontsize{#3}{1pt}\selectfont 
}{
  \end{adjustwidth}
}

\title{Notación y Diferenciación}
\date{l0-11-2025}
\author{Velázquez Ramírez Carlos Raúl}

\begin{document}
\pagenumbering{gobble}
  \maketitle
  \newpage
  \pagenumbering{roman}

\section*{Notación de la derivada}

Si $y = f(x)$, su derivada en el punto $x_0$ se denota por 

\[
  y'(x_0) = f'(x_0) = \frac{dy}{dx} |_{x = x_0} = \frac{df}{dx} |_{x = x_0} = D(x_0)
\]

Si la derivada no se evalúa, simplemente 

\[
  y'= f'(x) = \frac{dy}{dx} = \frac{df}{dx} = D(x)
\]

Derivación de orden superior: 

\[
  y'' = f''(x) = \frac{d^2 y}{d x^2}
\]

\textbf{Exampli gratia;} 

1) Muestre que $f$ es diferenciable. 

\[
  f: \mathbb{R} \to \mathbb{R} 
\]

\[
  x |\to c \text{, $c$ es fijo}
\]

Demostración: 

Sea $f(x) = c$, entonces $\forall x \in \mathbb{R} \setminus \{ x_0 \}$.

\[
  \frac{f(x) - f(x_0)}{x - x_0} = \frac{c - c}{x - x_0} = 0
\]

Entonces 

\[
  \lim\limits_{x \to x_0} \frac{f(x) - f(x_0)}{x - x_0} = \lim\limits_{x \to x_0} 0 = 0
\]

\[
  \therefore f \text{ es diferenciable}
\]

2) Muestre que $f: \mathbb{R} \to \mathbb{R}$ dado por $f(x) = x^n$, $n \in \mathbb{N}$ es diferenciable en $\mathbb{R}$. 

Demostración: Sean $x, \, x_0 \in \mathbb{R}$ con $x \neq x_0$. 

\[
  \frac{f(x) - f(x_0)}{x - x_0} = \frac{x^n - x^n_0}{x - x_0} = \frac{(x - x_0) (x^{n - 1} + x^{n - 2} x + \dots + x x^{n - 2}_0 + x^{n - 1}_0)}{(x - x_0)}
\]

\[
  = x^{n - 1} + x^{n - 2} x + \dots + x x^{n - 2}_0 + x^{n - 1}_0
\]

Debido a que 

\[
  x^1 - y^1 = (x - y) \cdot 1
\]

\[
  x^2 - y^2 = (x - y) (x + y)
\]

\[
  x^3 - y^3 = (x - y) (x^2 + xy + y^2)
\]

\[
  x^4 - y^4 = (x - y) (x + y) (x^2 + y^2) = (x - y) (x^3 + x^2 y + x y^2 + y^3)
\]
\begin{center}
\dots\dots\dots
\end{center}
\[
  x^n - y^n = (x - y) (x^{n - 1} + x^{n - 2} y + x^{n - 3} y^2 + \dots + x^2 y^{n -3} + x y^{n-2} + y^{n-1})
\]

Entonces 

\[
  \lim\limits_{x \to x_0} \frac{f(x) - f(x_0)}{x - x_0} = n x^{n - 1}_0
\]

Así, $f$ es diferenciable y $f'(x) = n x^{x - 1} \forall x \in \mathbb{R}$.
\end{document}
