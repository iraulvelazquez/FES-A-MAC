\documentclass{article} % Document type 
\usepackage[spanish]{babel} % Language typesetting 
\usepackage[utf8]{inputenc} % Accented chars without commands
\usepackage[T1]{fontenc} % Accented display corrected
\selectlanguage{spanish} % Default lang, useful for for multi lang switching 
\usepackage{amsmath} % Math 1/3
\usepackage{amssymb} % Math 2/3
\usepackage{stmaryrd} % Math 3/3
\usepackage{tikz} % Crating graphics programmatically
\usepackage{pgfplots} % Extends tikz for 2D and 3D plots for data and math f 
\pgfplotsset{compat=1.18} % ---
\usepackage{circuitikz} % Extends tikz for electrical circuits
\usepackage[dvipsnames, svgnames]{xcolor} % Coloring
\usepackage{soul} % Text highlighting 
\setul{0.5ex}{0.3ex} % ---
\setlength{\parindent}{15pt} % Controls indentantion for the first line of paragraphs 
\setlength{\parskip}{0.8\baselineskip plus 0.2\baselineskip minus 0.1\baselineskip} % Vertical spacing between paragraphs 
\usepackage{geometry} % Layout
\geometry{margin=3cm} % --- 

\title{Límites laterales}
\date{17-10-2025}
\author{Velázquez Ramírez Carlos Raúl}

\begin{document}
\pagenumbering{gobble}
  \maketitle
  \newpage
  \pagenumbering{roman}

\textbf{Exampli gratia:} 

1) $\lim\limits_{x \to 0} \frac{1}{x}$.

\[
\lim\limits_{x \to 0} \frac{1}{x}=\frac{1}{0}=\infty
\]

\begin{center}
\begin{tikzpicture}
  \begin{axis}
    [
      axis lines = middle,
      xlabel = $x$,
      ylabel = $y$,
      xmax = 5,
      xmin = -5,
      ymax = 5,
      ymin = -5,
    ]
    \addplot
      [
        domain=-5:5,
        samples=100,
        color=blue,
      ] 
    {1/x};
    \addlegendentry{$\frac{1}{x}$}
  \end{axis}
\end{tikzpicture}
\end{center}

Limite por izq:

\[
  \lim\limits_{x \to 0^-} \frac{1}{x}=\lim\limits_{x \to 0} \frac{1}{x}=-\infty, x<0
\]

Limite por la derecha:

\[
  \lim\limits_{x \to 0^+} \frac{1}{x}=\lim\limits_{x \to o} \frac{1}{x}=+\infty, x>0
\] 

Menos infinito:

\[
  \lim\limits_{x \to -\infty} \frac{1}{x}=\frac{1}{-\infty}=0
\] 

Más infinito:

\[
  \lim\limits_{x \to +\infty} \frac{1}{x}=\frac{1}{+\infty}=0
\] 

2) Estudiar $\lim\limits_{x \to 0} sng x$.

\[sgn(x)= 
  \begin{cases}
    1 & \text{si $x>0$}\\ 
    0 & \text{si $x=0$}\\ 
    -1 & \text{si $x<0$}
  \end{cases}
\]

\begin{center}
\begin{tikzpicture}
  \begin{axis}
    [
      axis lines = middle,
      xlabel = $x$,
      ylabel = $y$,
      xmax = 5,
      xmin = -5,
      ymax = 5,
      ymin = -5,
    ]
    \addplot
      [
        domain=-5:0,
        samples=100,
        color=blue,
      ] 
    {-1};
    \addlegendentry{$f(x)=-1$}
    \addplot
       [
         domain=0:5,
         samples=100,
         color=blue,
       ] 
     {1};
     \addlegendentry{$f(x)=1$}
    \addplot
      [
       color=blue,
       only marks,
       mark=*,
       mark size=2pt,
       forget plot,
      ]
    coordinates {(0,0)};

  \end{axis}
\end{tikzpicture}
\end{center}

Limite por izq:

\[
  \lim\limits_{x \to 0^-} sng x=\lim\limits_{x \to 0} (-1)=-1
\]

Limite por der:

\[
  \lim\limits_{x \to 0^+} sng x=\lim\limits_{x \to 0} (1)=1
\]

Como ambos, el límite por izq. y por der. no coinciden, concluimos que el limite no existe.

3) Calcule $\lim\limits_{x \to 0}\frac{1}{x^2}$.

\[
\lim\limits_{x \to 0}\frac{1}{x^2}=\frac{1}{0}=\infty 
\]

Como el denominador es $x^2$, este $\infty$ es «forzado»a ser positivo, aunque es válido percibirlo por casos.

\begin{center}
\begin{tikzpicture}
  \begin{axis}
    [
      axis lines = middle,
      xlabel = $x$,
      ylabel = $y$,
      xmax = 5,
      xmin = -5,
      ymax = 5,
      ymin = -5,
    ]
    \addplot
      [
        domain=-5:5,
        samples=100,
        color=blue,
      ] 
      {1/x^2};
    \addlegendentry{$\frac{1}{x^2}$}
  \end{axis}
\end{tikzpicture}
\end{center}

4) Estudie la continuidad de $f$ en los puntos de ruptura.

\[f(x)= 
  \begin{cases}
    x^2-1 & \text{, si $x \leq 0$}\\ 
    \frac{1}{x} & \text{, si $x>0$}\\ 
  \end{cases}
\]

\begin{center}
\begin{tikzpicture}
  \begin{axis}
    [
      axis lines = middle,
      xlabel = $x$,
      ylabel = $y$,
      xmax = 10,
      xmin = -10,
      ymax = 10,
      ymin = -10,
    ]
    \addplot
      [
        domain=-10:1,
        samples=100,
        color=blue,
      ] 
    {x^2-1};
    \addlegendentry{$x^2-1$}
  \addplot
    [
      domain=1:10,
      samples=100,
      color=blue,
    ] 
  {1/x};
  \addlegendentry{$\frac{1}{x}$}
  \addplot
    [
      color=black,
      only marks,
      mark=*,
      mark size=2pt,
      forget plot,
    ]
  coordinates {(1,0)};
  \addplot
    [
      color=black,
      only marks,
      mark=o,
      mark size=2pt,
      forget plot,
    ]
  coordinates {(1,1)};
  \end{axis}
\end{tikzpicture}
\end{center}

¿$f$ es continua en $x=1$? 

Por un lado, 

\[
f(1)=1^2-1=0 
\]

Por otro lado, calculemos $\lim\limits_{x \to 1} f(x)$. Para esto calculamos limites laterales, 

\[
\lim\limits_{x \to 1^+} f(x)=\lim\limits_{x \to 1} \frac{1}{x}=1,\, x>1
\] 

\[
\lim\limits_{x \to 1^-} f(x)=\lim\limits_{x \to 1} (x^2-1)=0,\, x<1
\] 

Como los limites laterales son distintos, el limite no existe y, por lo tanto, $f$ es discontinua en $x=1$.

5) Estudie la continuidad de $f(x)=mx+b$, $m,b \in \mathbb{R}$.

\[
Dom_f=\mathbb{R}
\]

Sea $a \in \mathbb{R}$. Veamos si $f$ es continua en $x=a$,

Por un lado, 

\[
f(x)=ma+b
\]

Por otro lado,

\[
\lim\limits_{x \to a} f(x)=ma+b 
\]

Como $f(x)=\lim\limits_{x \to a}$, y $a \in \mathbb{R}$ es arbitrario, concluimos que $f(x)=mx+b$ es continua en $\mathbb{R}$.

\textit{Otras funciones que son continuas en su dominio:}

i) Funciones polinómicas.

ii) Funciones racionales. 

6) Determine si $f(x)=|2x-1|$ es continua en $\mathbb{R}$. 

Solución: 

\[f(x)= 
  \begin{cases}
    2x-1 & \text{, si $x\geq\frac{1}{2}$}\\ 
    -(2x-1) & \text{, si $x<\frac{1}{2}$}\\ 
  \end{cases}
\]

Como las rectas son funciones polinómicas, $f(x)$ es continua en $\mathbb{R} \setminus \{\frac{1}{2}\}$.

Ahora, estudiemos la continuidad en $x=\frac{1}{2}$.

Por un lado, 

\[
  f(\frac{1}{2})=2(\frac{1}{2})-1=0 
\]

Por otro lado, calculemos $\lim\limits_{x \to \frac{1}{2}} f(x)$,

\[
  \lim\limits_{x \to \frac{1}{2}^-} f(x)=\lim\limits_{x \to \frac{1}{2}}-(2x-1)=-(2\cdot\frac{1}{2}-1)=0, x<\frac{1}{2}
\]

\[
  \lim\limits_{x \to \frac{1}{2}^+} f(x)=\lim\limits_{x \to \frac{1}{2}}(2x-1)=2\cdot\frac{1}{2}-1=0, x>\frac{1}{2}
\]

Como los límites laterales coinciden, el límite existe. Además, como $\lim\limits_{x \to \frac{1}{2}} f(x)=f(\frac{1}{2})$, $f$ es continua en $x=\frac{1}{2}$. Por lo tanto, $f$ es continua en $\mathbb{R}$.

7) $y=|\frac{1}{x}|$.

\[|\frac{1}{x}|= 
  \begin{cases}
    \frac{1}{x} & \text{, si $x>0$}\\ 
    -\frac{1}{x} & \text{, si $x<0$}
  \end{cases}
\]

\begin{center}
\begin{tikzpicture}
  \begin{axis}
    [
      axis lines = middle,
      xlabel = $x$,
      ylabel = $y$,
      xmax = 5,
      xmin = -5,
      ymax = 5,
      ymin = -5,
    ]
    \addplot
      [
        domain=-5:5,
        samples=100,
        color=blue,
      ] 
    {abs(1/x)};
    \addlegendentry{$|\frac{1}{x}|$}
  \end{axis}
\end{tikzpicture}
\end{center}

8) $\lim\limits_{x \to 6}\frac{\sqrt{6}-\sqrt{x}}{36-x^2}$.

\[
  \lim\limits_{x \to 6}\frac{\sqrt{6}-\sqrt{x}}{36-x^2}=\frac{\sqrt{6}-\sqrt{6}}{36-36}=\frac{0}{0} \text{ , indeterminación}
\]

\[
\lim\limits_{x \to 6}\frac{\sqrt{6}-\sqrt{x}}{(6-x)(6+x)}=\lim\limits_{x \to 6}\frac{\sqrt{6}-\sqrt{x}}{(\sqrt{6}-\sqrt{x})(\sqrt{6}+\sqrt{x})(6+x)}=\lim\limits_{x \to 6}\frac{1}{(\sqrt{6}+\sqrt{x})(6+x)}
\]
\[
=\frac{1}{(\sqrt{6}+\sqrt{6})(6+6)}=\frac{1}{(2\sqrt{6})12}=\frac{1}{24\sqrt{6}}\cdot\frac{\sqrt{6}}{\sqrt{6}}=\frac{6}{144}
\]
\[
  x \neq 6 \text{, por $\sqrt{6}-\sqrt{x}$}
\]

9) $\lim\limits_{x \to 5}=\frac{\sqrt{x+4}-3}{x-5}$.


\[
  \lim\limits_{x \to 5}=\frac{\sqrt{x+4}-3}{x-5}=\frac{3-3}{5-5}=\frac{0}{0} \text{, indeterminación}
\]

\[
  \lim\limits_{x \to 5}=\frac{\sqrt{x+4}-3}{x-5}\cdot\frac{\sqrt{x+4}+3}{\sqrt{x+4}+3}=\lim\limits_{x \to 5}\frac{(\sqrt{x+4})^2-3^2}{(x-5)(\sqrt{x+4}+3)} 
\]

\[
\text{, recordemos que $(\sqrt{a}-\sqrt{b})(\sqrt{a}+\sqrt{b})=a^2-b^2$}
\]

\[
 =\lim\limits_{x \to 5}\frac{x+4-9}{(x-5)(\sqrt{x+4}+3)}=\lim\limits_{x \to 5}\frac{x-5}{(x-5)(\sqrt{x+4}+3)}=\lim\limits_{x \to 5}\frac{1}{\sqrt{x+4}+3}=\frac{1}{6}
\]

\[
  x \neq 5 \text{, por $x-5$} 
\]

Pero gráficamente,

\begin{center}
\begin{tikzpicture}
  \begin{axis}
    [
      axis lines = middle,
      xlabel = $x$,
      ylabel = $y$,
      xmax = 5,
      xmin = -5,
      ymax = 5,
      ymin = -5,
    ]
    \addplot
      [
        domain=-5:5,
        samples=100,
        color=blue,
      ] 
    {sqrt(x+4)};
    \addlegendentry{$\sqrt{x+4}$}
    \addplot
      [
        domain=-5:5,
        samples=100,
        color=blue,
      ]
    {-3};
    \addlegendentry{$y+3$}
  \end{axis}
\end{tikzpicture}
\end{center}

, no existe intersección. Por lo tanto, solo hay solución en $\mathbb{C}$.
\end{document}
