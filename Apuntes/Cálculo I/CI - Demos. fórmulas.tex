\documentclass{article} % Document type 
\usepackage[spanish]{babel} % Language typesetting 
\usepackage[utf8]{inputenc} % Accented chars without commands
\usepackage[T1]{fontenc} % Accented display corrected
\selectlanguage{spanish} % Default lang, useful for for multi lang switching 
\usepackage{amsmath} % Math 1/3
\usepackage{amssymb} % Math 2/3
\usepackage{stmaryrd} % Math 3/3
\usepackage{tikz} % Crating graphics programmatically
\usepackage{pgfplots} % Extends tikz for 2D and 3D plots for data and math f 
\pgfplotsset{compat=1.18} % ---
\usepackage{circuitikz} % Extends tikz for electrical circuits
\usepackage[dvipsnames, svgnames]{xcolor} % Coloring
\usepackage{soul} % Text highlighting 
\setul{0.5ex}{0.3ex} % ---
\setlength{\parindent}{15pt} % Controls indentantion for the first line of paragraphs 
\setlength{\parskip}{0.8\baselineskip plus 0.2\baselineskip minus 0.1\baselineskip} % Vertical spacing between paragraphs 
\usepackage{geometry} % Layout
\geometry{margin=3cm} % --- 
\usepackage{titlesec} % Global section centering
\titleformat{\section} % --
  {\normalfont\Large\bfseries\centering}{\thesection}{1em}{} % --
\usepackage{changepage} % Special paragraph
\newenvironment{x}[3]{
  \begin{adjustwidth}{#1}{0cm}
  \setlength{\parindent}{0pt}
  #2\fontsize{#3}{1pt}\selectfont 
}{
  \end{adjustwidth}
}

\title{Formulas, Demostraciones y Derivación Implícita}
\date{14-11-2025}
\author{Velázquez Ramírez Carlos Raúl}

\begin{document}
\pagenumbering{gobble}
  \maketitle
  \newpage
  \pagenumbering{roman}

\section*{Demostraciones}

Consideremos un cambio de variable, 

\[
  f'(x_0) = \lim\limits_{x \to x_0} \frac{f(x) - f(x_0)}{x - x_0} 
\]

Sea $h = x - x_0$, entonces 

\[
  \lim\limits_{x \to x_0} \frac{f(x) - f(x_0)}{x - x_0} = \lim\limits_{h \to 0} \frac{f(x_0 + h) - f(x_0)}{h}
\]

Si $x \to x_0$, entonces $h \to 0$.

1) Demostración de $\frac{d}{dx} \sin(x) = \cos(x)$.

\[
  \frac{f(x_0 + h) - f(x_0)}{h} = \frac{\sin(x_0 + h) - \sin(x_0)}{h}
\]

Recordemos que $\sin(A + B) = \sin(A) \cos(B) + \sin(B) \cos(A)$, 

\[
  = \frac{\sin(x_0) \cos(h) + \sin(h) \cos(x_0) - \sin(x_0)}{h} = \frac{\sin(x_0) (\cos(h) - 1) + \sin(h) \cos(x_0)}{h}
\]

\[
  = \frac{\sin(x_0) (\cos(h) - 1)}{h} \cdot \frac{\cos(h) + 1}{\cos(h) +1} + \frac{\sin(h)}{h} \cos(x_0)
\]

\[
  \frac{- \sin(x_0) \sin^2(h)}{h (\cos(h) + 1)} + \frac{\sin(h)}{h} \cdot \cos(x_0)
\]

\[
  = \frac{\sin(h)}{h} [\frac{- \sin(x_0) \sin(h)}{\cos(h) + 1}] + \frac{\sin(h)}{h} \cdot \cos(x)
\]

Entonces, 

\[
  \lim\limits_{h \to 0} \frac{f(x_0 + h) - f(x_0)}{h} = \lim\limits_{h \to 0} \frac{\sin(h)}{h} \cdot \lim\limits_{h \to 0} \frac{- \sin(x_0) \sin(h)}{\cos(h) + 1} + \lim\limits_{h \to 0} \frac{\sin(h)}{h} \cdot \lim\limits_{h \to 0} \cos(x)
\]

\[
  = (1)(0) + (1)(\cos(x_0)) = \cos(x_0)
\]

\section*{Funciones polinómicas} 

\[
  \frac{d}{dx} c = 0
\]

\[
  \frac{d}{dx} x = 1
\]

\[
  \frac{d}{dx} x^n = n x^{n - 1}, \, n \in \mathbb{R}
\]

\[
  \frac{d}{dx} v^n = n v^{n - 1} \cdot \frac{dv}{dx}
\]

\[
  \frac{d}{dx} (u \pm v) = \frac{du}{dx} \pm \frac{dv}{dx}
\]

\[
  \frac{d}{dx} (cu) = c \frac{du}{dx}
\]

\[
  \frac{d}{dx} (f \circ g)(x) = f'(g(x)) \cdot g'(x)
\]

\[
  \frac{d}{dx} (uv) = uv' + u'v
\]

\[
  \frac{d}{dx} (\frac{u}{v}) = \frac{vu' - uv'}{v^2} 
\]

Donde $u$ y $v$ son funciones y $c$ es una constante.


\textbf{Exampli gratia:} 

1) 

\[
  y = 7 \sqrt{x} - \frac{2}{\sqrt{x}} + \frac{3}{4} \sqrt[3]{x^5} = 7 x^{\frac{1}{2}} - 2 x^{-\frac{1}{2}} + \frac{3}{4} x^{\frac{5}{3}}
\]

\[
  y' = 7 \cdot \frac{1}{2} x^{- \frac{1}{2}} - 2 \cdot (- \frac{1}{2}) x^{- \frac{3}{2}} + \frac{3}{4} \cdot \frac{5}{3} x^{\frac{2}{3}} = \frac{7}{2 \sqrt{x}} + \frac{1}{\sqrt{x^3}} + \frac{5}{4} \sqrt[3]{x^2}
\]

2) 

\[
  y = \sqrt[\pi]{9} - \sqrt[\sqrt{2}]{4} x
\]

\[
  y' = \sqrt[\sqrt{2}]{4} \cdot 1 = \sqrt[\sqrt{2}]{4}
\]

3) 

\[
  y = (2x - 3)^{2027}
\]

\[
  y' = 2027 (2x - 3)^{2026} \cdot 2 = 4054 (2x - 3)^{2026}
\]

4) 

\[
  y = \cos(x^2) 
\]

\[
  y' = (- \sin(x^2)) 2x = -2x \sin(x^2)
\]

5) 

\[
  y = \tan(x) = \frac{\sin(x)}{\cos(x)} 
\]

\[
  y' = \frac{\cos(x) \cos(x) - \sin(x) (- \sin(x))}{\cos^2(x)} = \frac{\cos^2(x) + \sin^2(x)}{\cos^2(x)} = \frac{1}{\cos^2(x)} 
\]

\[
  = (\frac{1}{\cos^2(x)})^2 = (\sec(x))^2 = \sec^2(x)
\]
6) 

\[
  y = \sqrt{4x - 3} = (4x - 3)^{\frac{1}{2}}
\]

\[
  y' = \frac{1}{2} (4x - 3)^{- \frac{1}{2}} \cdot 4 = \frac{2}{\sqrt{4x - 3}}
\]

7) 

\[
  y = (8x - 3)^5 (3x + 7)^8
\]

\[
  u = (8x - 3)^5 \Rightarrow u' = 5 (8x - 3)^4 \cdot 8 = 40 (8x - 3)^4
\]

\[
  v = (3x + 7)^8 \Rightarrow v' = 8 (3x + 7)^7 \cdot 3 = 24 (3x + 7)^7
\]

\[
  y' = (8x - 3)^5 \cdot 24 (3x + 7)^7 + 40 (8x - 3)^4 \cdot (3x + 7)^8 = 8 (8x - 3)^4 (3x + 7)^7 [3 (8x - 3) + 5 (3x + 7)]
\]

\[
  = 8 (8x - 3)^4 (3x + 7)^7 (39x + 26) = 104 (8x - 3)^4 (3x + 7)^7 (3x + 2)
\]

\textbf{Nota:} Grandville y Spivak para estas demostraciones.

\section*{Funciones trigonométricas}

\[
  \frac{d}{dx} \sin(x) = \cos(x)
\]

\[
  \frac{d}{dx} \sin(v) = v' \cos(v)
\]

\[
  \frac{d}{dx} \cos(x) = - \sin(x)
\]

\[
  \frac{d}{dx} \cos(v) = -v' \sin(v)
\]

\[
  \frac{d}{dx} \tan(x) = \sec^2(x)
\]

\[
  \frac{d}{dx} \tan(v) = v' \sec^2(v)
\]

\[
  \frac{d}{dx} \cot(x) = - \csc^2(x)
\]

\[
  \frac{d}{dx} \cot(v) = - v' \csc^2(v)
\]

\[
  \frac{d}{dx} \sec(x) = \sec(x) \tan(x)
\]

\[
  \frac{d}{dx} \sec(v) = v' \sec(v) \tan(v)
\]

\[
  \frac{d}{dx} \csc(x) = - \csc(x) \cot(x)
\]

\[
  \frac{d}{dx} \csc(v) = - v' \csc(v) \cot(v)
\]

Donde $v$ es el argumento.

\section*{Funciones logaritmicas}

\[
  \frac{d}{dx} e^x = e^x
\]

\[
  \frac{d}{dx} e^v = v' e^v
\]

\[
  \frac{d}{dx} \ln(x) = \frac{1}{x} 
\]

\[
  \frac{d}{dx} \ln(v) = \frac{v'}{v}
\]

8) 

\[
  y = \sin(\sqrt{x}) 
\]

\[
  y' = \frac{1}{2 \sqrt{x}} \cdot \cos{\sqrt{x}} = \frac{\cos(\sqrt{x})}{2 \sqrt{x}}
\]

9) 

\[
  y = e^{- \frac{1}{3} x}
\]

\[
  y' = - \frac{1}{3} e^{- \frac{1}{3} x}
\]

10) 

\[
  y = \ln (\cos(x)) 
\]

\[
  y' = \frac{1}{\cos(x)} \cdot (- \sin(x)) = - \frac{\sin(x)}{\cos(x)} = - \tan(x)
\]

11) 

\[
  y = \ln(\ln(\ln(\ln(x)))) 
\]

\[
  y' = \frac{[\ln(\ln(\ln(x)))]'}{\ln(\ln(\ln(x)))} = \frac{1}{\ln(\ln(\ln(x)))} \cdot \frac{[\ln(\ln(x))]'}{\ln(\ln(x))} = \frac{1}{\ln(\ln(\ln(x)))} \cdot \frac{1}{\ln(\ln(x))} \cdot \frac{(\ln(x))'}{\ln(x)} 
\]

\[
  = \frac{1}{\ln(\ln(\ln(x)))} \cdot \frac{1}{\ln(\ln(x))} \cdot \frac{1}{\ln(x)} \cdot \frac{1}{x}
\]

Recordemos que $a \ln(x) = \ln(x^a)$,

\[
  y' = \frac{1}{x \ln(x) \ln(\ln(x)) \ln(\ln(\ln(x)))} = \frac{1}{\ln(\ln(\ln(x)))^{\ln(\ln(x))^{\ln(x^x)}}}
\]

\section*{Derivación implícita} 

\[
  x^2 + y^2 = 1 \to y = \sqrt{1 - x^2} = (1 - x^2)^{\frac{1}{2}} \Rightarrow y' = \frac{1}{2} (1 - x^2)^{- \frac{1}{2}} \cdot (-2x) = \frac{-x}{\sqrt{1 - x^2}}
\]

La derivación implícita requiere de derivar con la expresión \textit{en crudo}.

\textbf{Exampli gratia:} Derivemos ambos miembros de la ecuación $x^2 + y^2 = 1$, 

\[
  \frac{d}{dx} (x^2 + y^2) = \frac{d}{dx} (1) 
\]

\[
  \frac{d}{dx} x^2 + \frac{d}{dx} y^2 = 0
\]

\[
  2x + 2y \cdot y' = 0
\]

\[
  2y y' = - 2x 
\]

\[
  y' = \frac{- 2x}{2y} = - \frac{x}{y}
\]

Pero $y = \sqrt{1 - x^2}$, 

\[
  y' = \frac{-x}{\sqrt{1 - x^2}}
\]

\section*{Demostraciones de derivación implícita} 

\textbf{Exampli gratia:} 

1) Demostremos que $\frac{d}{dx} e^x = e^x$, 

Sea $y = e^x$, 

\[
  \ln(y) = \ln(e^x)
\]

\[
  \ln(y) = x \ln(e)
\]

\[
  \ln(y) = x \cdot 1
\]

\[
  \ln(y) = x
\]

Derivamos implícitamente: 

\[
  \frac{d}{dx} (\ln(y)) = \frac{d}{dx} x
\]

\[
  \frac{y'}{y} = 1
\]

\[
  y' = y
\]

\[
  y' = e^x
\]

2) $\frac{d}{dx} a^v = a^v \ln(a) \cdot v'$.

Demostración: 

Sea $y = a^v$, 

\[
  \ln(y) = \ln(a^v) 
\]

\[
  \ln(y) = v \ln(a)
\]

\[
  \frac{d}{dx} \ln(y) = \frac{d}{dx} (\ln(a) \cdot v)
\]

\[
  \frac{y'}{y} = \ln(a) \cdot v'
\]

\[
  y' = \ln(a) v' y
\]

\[
  y' = \ln(a) v' a^v
\]
\end{document}
