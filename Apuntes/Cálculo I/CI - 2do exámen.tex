\documentclass{article} % Document type 
\usepackage[spanish]{babel} % Language typesetting 
\usepackage[utf8]{inputenc} % Accented chars without commands
\usepackage[T1]{fontenc} % Accented display corrected
\selectlanguage{spanish} % Default lang, useful for for multi lang switching 
\usepackage{amsmath} % Math 1/3
\usepackage{amssymb} % Math 2/3
\usepackage{stmaryrd} % Math 3/3
\usepackage{tikz} % Crating graphics programmatically
\usepackage{pgfplots} % Extends tikz for 2D and 3D plots for data and math f 
\pgfplotsset{compat=1.18} % ---
\usepackage{circuitikz} % Extends tikz for electrical circuits
\usepackage[dvipsnames, svgnames]{xcolor} % Coloring
\usepackage{soul} % Text highlighting 
\setul{0.5ex}{0.3ex} % ---
\setlength{\parindent}{15pt} % Controls indentantion for the first line of paragraphs 
\setlength{\parskip}{0.8\baselineskip plus 0.2\baselineskip minus 0.1\baselineskip} % Vertical spacing between paragraphs 
\usepackage{geometry} % Layout
\geometry{margin=3cm} % --- 

\title{Segundo Parcial}
\date{10-11-2025}
\author{Velázquez Ramírez Carlos Raúl}

\begin{document}
\pagenumbering{gobble}
  \maketitle
  \newpage
  \pagenumbering{roman}

\section{Calcule:}

a) $\lim\limits_{x \to -2} \frac{x^2 - 4}{1 - \sqrt{x + 3}} = \frac{0}{0}$, indeterminación.

\[
  \lim\limits_{x \to -2} \frac{x^2 - 4}{1 - \sqrt{x + 3}} \cdot \frac{1 + \sqrt{x + 3}}{1 + \sqrt{x + 3}} = \lim\limits_{x \to -2} \frac{(x - 2) (x + 2) (1 + \sqrt{x + 3})}{-(x + 2)} 
\]

\[
  = \lim\limits_{x \to -2} (2 - x)(1 + \sqrt{x + 3}) = 4(2) = 8
\]

b) $\lim\limits_{x \to 0} \frac{x - \sin(3x)}{x + \sin(5x)} = \frac{0}{0}$, indeterminación.

\[
  \lim\limits_{x \to 0} \frac{x - \sin(3x)}{x + \sin(5x)} \cdot \frac{\frac{1}{x}}{\frac{1}{x}} = \lim\limits_{x \to 0} \frac{1 - \frac{3 \sin(3x)}{3x}}{1 + \frac{5 \sin(5x)}{5x}}  = \frac{\lim\limits_{x \to 0} 1 - 3 \lim\limits_{x \to 0} \frac{\sin(3x)}{3x}}{\lim\limits_{x \to 0} 1 + 5 \lim\limits_{x \to 0} \frac{\sin(5x)}{5x}} = \frac{1 - 3 \cdot 1}{1 + 5 \cdot 1} = - \frac{2}{6} = - \frac{1}{3}
\]

c) $\lim\limits_{x \to 0} \frac{\tan(x) - \sin(x)}{x^3} = \frac{0}{0}$, indeterminación.

\[
  \lim\limits_{x \to 0} \frac{\frac{\sin(x)}{\cos(x)} - \frac{\sin(x)}{1}}{x^3} = \lim\limits_{x \to 0} \frac{\frac{\sin(x) - \sin(x) \cos(x)}{\cos(x)}}{\frac{x^3}{1}} = \lim\limits_{x \to 0} \frac{\sin(x) (1 - \cos(x))}{x^3 \cos(x)} \cdot \frac{1 + \cos(x)}{1 + \cos(x)}
\]

\[
  = \lim\limits_{x \to 0} \frac{\sin(x) (1 - \cos^2(x))}{x^3 \cos(x)(1 + \cos(x))} = \lim\limits_{x \to 0} \frac{\sin^3(x)}{x^3} \cdot \frac{1}{\cos(x) + \cos^2(x)} 
\]

\[
  = \lim\limits_{x \to 0} \frac{\sin(x)}{x} \cdot \lim\limits_{x \to 0} \frac{\sin(x)}{x} \cdot \lim\limits_{x \to 0} \frac{\sin(x)}{x} \cdot \lim\limits_{x \to 0} \frac{1}{\cos(x) + \cos^2(x)}
\]

\[
  = \lim\limits_{x \to 0} 1 \cdot 1 \cdot 1 \cdot \frac{1}{1 + 1^2} = \frac{1}{2}
\]

d) $\lim\limits_{x \to \infty} \frac{\sqrt{2x^2 + 1} - x}{\sqrt{x^2 + x} + \sqrt{x^2 + 2}} = \frac{\infty - \infty}{\infty}$, indeterminación.

\[
  \lim\limits_{x \to \infty} \frac{\sqrt{2x^2 + 1} - x}{\sqrt{x^2 + x} + \sqrt{x^2 + 2}} \cdot \frac{\sqrt{2x^2 + 1} + x}{\sqrt{2x^2 + 1} + x} = \lim\limits_{x \to \infty} \frac{x^2 + 1}{(\sqrt{x^2 + x} + \sqrt{x^2 + 2}) (\sqrt{2x^2 + 4} + x)} = \frac{\infty}{\infty} 
\]
\[
  \text{, indeterminación}
\]

\[
  \lim\limits_{x \to \infty} \frac{1 + \frac{1}{x^2}}{\frac{1}{x} (\sqrt{x^2 + x} + \sqrt{x^2 + 2}) \frac{1}{x}(\sqrt{2x^2 + 1} + x)} = \lim\limits_{x \to \infty} \frac{1 + \frac{1}{x^2}}{(\sqrt{1 + \frac{1}{x^2}} + \sqrt{1 + \frac{2}{x^2}}) (\sqrt{2 + \frac{1}{x^2}} + 1)}
\]

\[
  = \frac{1}{(\sqrt{1} + \sqrt{1})(\sqrt{2} + 1)} = \frac{1}{2 (\sqrt{2} + 1)} \cdot \frac{\sqrt{2} - 1}{\sqrt{2} - 1} = \frac{\sqrt{2} - 1}{2}
\]

Un método más simple sería: 

\[
  \lim\limits_{x \to \infty} \frac{\sqrt{2x^2 + 1} - x}{\sqrt{x^2 + x} + \sqrt{x^2 + 2}} \cdot \frac{\frac{1}{x}}{\frac{1}{x}} = \lim\limits_{x \to \infty} \frac{\sqrt{2 + \frac{1}{x^2}} - 1}{\sqrt{1 + \frac{1}{x}} + \sqrt{1 + \frac{2}{x^2}}} = \frac{\sqrt{2} - 1}{2}
\]

\section{Grafique y estudie la continuidad de la función en $\mathbb{R}$:}

\[f(x) 
  \begin{cases}
    \frac{|x - 3|}{x - 3} & \text{, si } x \neq 3\\ 
    1 & \text{, si } x = 3
  \end{cases}
\]

\[|x - 3|
  \begin{cases}
    x - 3 & \text{, si } x \geq 3\\ 
    -(x - 3) & \text{, si } x < 3
  \end{cases}
\]

\[f(x)
  \begin{cases}
    \frac{-(x - 3)}{x - 3} & \text{, si } x < 3\\ 
    1 & \text{, si } x = 3\\ 
    \frac{x - 3}{x - 3} & \text{, si } x > 3
  \end{cases}
\]

\[f(x) 
  \begin{cases}
    -1 & \text{, si } x < 3\\ 
    1 & \text{, si } x \geq 3
  \end{cases}
\]

\begin{center}
\begin{tikzpicture}
  \begin{axis}
    [
      axis lines = middle,
      xlabel = $x$,
      ylabel = $y$,
      xmax = 10,
      xmin = -10,
      ymax = 4,
      ymin = -4,
    ]
    \addplot
      [
        domain=3:10,
        samples=100,
        color=blue,
      ] 
    {1};
    \addplot
      [
        domain=-10:3,
        samples=100,
        color=blue,
      ] 
    {-1};
    \addplot
      [
       color=black,
       only marks,
       mark=*,
       mark size=2pt,
       forget plot,
      ]
    coordinates {(3,1)};
    \addplot
      [
       color=black,
       only marks,
       mark=o,
       mark size=2pt,
       forget plot,
      ]
    coordinates {(3,-1)};
  \end{axis}
\end{tikzpicture}
\end{center}

$f$ es continua en $\mathbb{R} \setminus \{ 3 \}$ por ser polinomio.

Para $x = 3$, estudiamos límites laterales. 

\[
  \lim\limits_{x \to 3^-} f(x) = \lim\limits_{x \to 3} (-1) = -1, \, x < 3
\]

\[
  \lim\limits_{x \to 3^+} f(x) = \lim\limits_{x \to 3} (1) = 1, \, x > 3
\]

Como $\lim\limits_{x \to 3^-} f(x) \neq \lim\limits_{x \to 3^+} f(x)$, $\lim\limits_{x \to 3} f(x)$ no existe y, por lo tanto, $f$ no es continua en $x = 3$.

\section{Sea $f(x) = x^3 + 1$. Pruebe o refute:} 
\subsection*{a) la función $f$ es biyectiva.} 
\subsection*{b) la función $f$ es impar.}

a) 

Definimos $f: \mathbb{R} \to \mathbb{R}$. 

Como $Im_f = \mathbb{R}$, $f$ es suprayectiva. 

Para la inyectividad, sean $x_1, x_2 \in \mathbb{R}$ tal que $f(x_1) = f(x_2)$.

\[
\Leftrightarrow x^3_1 + 1 = x^3_2 + 1 \Leftrightarrow x^3_1 = x^3_2 \Leftrightarrow \sqrt[3]{x^3_1} = \sqrt[3]{x^3_2} \Leftrightarrow x_1 = x_2
\]

Así, $f$ es inyectiva. 

$\therefore f$ es biyectiva.

b) 

$f$ no es impar. 

Supóngase lo contrario, entonces $f(-x) = - f(x)$. 

Calculamos: 

\[
  f(-x) = (-x^3) + 1 = -x^3 + 1
\]

\[
  -f(x) = -(x^3 + 1) = -x^3 - 1
\]

\[
  f(-x) \neq f(x)
\]

\section{Sea $f(x) = x^3 + 1$:}
\subsection*{a) Calcule su inversa $f^{-1} (x)$.} 
\subsection*{b) Compruebe aritméticamente y gráficamente $f(x)$ y $f^{-1} (x)$.} 

a) Sea $y = x^3 + 1$. 

\[
  x^3 + 1 = y 
\]

\[
  x^3 = y - 1
\]

\[
  x = \sqrt[3]{y - 1}
\]

\[
  f^{-1} (x) = \sqrt[3]{x - 1}
\]

b) 

\[
  (f \circ f^{-1})(x) = f(f^{-1}(x)) = f(\sqrt[3]{x - 1}) = (\sqrt[3]{x - 1}) + 1 = x - 1 + 1 = x
\]

\[
  (f^{-1} \circ f)(x) = f^{-1}(x^3 + 1) = \sqrt[3]{x^3 + 1 - 1} = \sqrt[3]{x^3} = x
\]

\[
  (f \circ f^{-1})(x) = (f^{-1} \circ f)(x) = x
\]

\begin{center}
\begin{tikzpicture}
  \begin{axis}[
    axis lines = middle,
    xlabel = $x$,
    ylabel = $y$,
      xmax = 10,
      xmin = -6,
      ymax = 10,
      ymin = -6,
    ]
    \addplot[
      domain=-6:10,
      samples=100,
      color=blue,
      ] {x^3 + 1};
    \addlegendentry{$x^3 + 1$}
    \addplot[
      domain=-6:10,
      samples=100,
      color=red,
      ] {sign(x - 1) * abs(x - 1)^(1/3)};
    \addlegendentry{$\sqrt[3]{x - 1}$}
    \addplot[
      domain=-6:10,
      samples=100,
      color=green,
      ] {x};
    \addlegendentry{$y = x$}
  \end{axis}
\end{tikzpicture}
\end{center}

\end{document}
