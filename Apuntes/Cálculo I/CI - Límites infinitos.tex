\documentclass{article} % Document type 
\usepackage[spanish]{babel} % Language typesetting 
\usepackage[utf8]{inputenc} % Accented chars without commands
\usepackage[T1]{fontenc} % Accented display corrected
\selectlanguage{spanish} % Default lang, useful for for multi lang switching 
\usepackage{amsmath} % Math 1/3
\usepackage{amssymb} % Math 2/3
\usepackage{stmaryrd} % Math 3/3
\usepackage{tikz} % Crating graphics programmatically
\usepackage{pgfplots} % Extends tikz for 2D and 3D plots for data and math f 
\pgfplotsset{compat=1.18} % ---
\usepackage{circuitikz} % Extends tikz for electrical circuits
\usepackage[dvipsnames, svgnames]{xcolor} % Coloring
\usepackage{soul} % Text highlighting 
\setul{0.5ex}{0.3ex} % ---
\setlength{\parindent}{15pt} % Controls indentantion for the first line of paragraphs 
\setlength{\parskip}{0.8\baselineskip plus 0.2\baselineskip minus 0.1\baselineskip} % Vertical spacing between paragraphs 
\usepackage{geometry} % Layout
\geometry{margin=3cm} % --- 
\usepackage{titlesec} % Global section centering
\titleformat{\section} % --
  {\normalfont\Large\bfseries\centering}{\thesection}{1em}{} % --

\title{Límites infinitos}
\date{24-10-2025}
\author{Velázquez Ramírez Carlos Raúl}

\begin{document}
\pagenumbering{gobble}
  \maketitle
  \newpage
  \pagenumbering{roman}

\textbf{Exampli gratia:} 

1) 

\[
  \lim\limits_{x \to \infty}\frac{2x^2+5x-1}{3x^2+6x+9}=\frac{\infty}{\infty}
\]

Dividir entre la $x$ de mayor potencia:

\[
  \lim\limits_{x \to \infty}\frac{\frac{2x^2}{x^2}+\frac{5x}{x^2}-\frac{1}{x^2}}{\frac{3x^2}{x^2}+\frac{6x}{x^2}+\frac{9}{x^2}}=\frac{2}{3} \text{, puesto que dividir $\frac{\infty}{\infty}$ es igual a 0}
\]

2) 

\[
  \lim\limits_{x \to \infty} \frac{1}{x}=0
\]

3) 

\[
  \lim\limits_{x \to \infty} \frac{x+4}{x^2+8x-1}=\frac{\infty}{\infty} \text{, indeterminación}
\]

\[
  \lim\limits_{x \to \infty} \frac{\frac{x}{x^2}+\frac{4}{x^2}}{\frac{x^2}{x^2}+\frac{8x}{x^2}-\frac{1}{x^2}}=\frac{0}{1}=0
\]

4) 

\[
  \lim\limits_{x \to \infty} \frac{x^3+8x+2}{x-4}=\frac{\infty}{\infty} \text{, indeterminación}
\]

\[
  \lim\limits_{x \to \infty} \frac{\frac{x^3}{x^3}+\frac{8x}{x^3}+\frac{2}{x^3}}{\frac{x}{x^3}-\frac{4}{x^3}}=\frac{1}{0}=\infty \text{, otra indeterminación}
\]

\[
  \lim\limits_{x \to \infty} \frac{x^3+8x+2}{x-4}=\lim\limits_{x \to infty} x^2+4x+24+\frac{98}{x-4}=\infty \text{, irresoluble}
\]

\textit{Resumen:} $\lim\limits_{x \to \infty} \frac{a_0+a_1x+\dots+a_nx^n}{b_0+b_1x+\dots+b_mx^m}$ puede tener:

\[\text{Casos:} 
  \begin{cases}
    \frac{a_n}{b_n}, & n=m\\ 
    \infty, & n>m\\ 
    0, & n<m
  \end{cases}
\]

\[
  \lim\limits_{x \to \infty} \frac{4x^6 + x^2 + x - 1}{8x^6 + 3x - 2} = \frac{4}{8} = \frac{1}{2}
\]

5)

\[
  \lim\limits_{x \to \infty} \frac{\sqrt{x^3+1}}{x+4}=\infty \text{, indeterminación}
\]

\[
  \lim\limits_{x \to \infty} \frac{\frac{\sqrt{x^3+1}}{\sqrt{x^3}}}{\frac{x+4}{\sqrt{x^3}}}=\lim\limits_{x \to \infty} \frac{\sqrt{\frac{x^3}{x^3}+\frac{1}{x^3}}}{\frac{x}{x^{3/2}}+\frac{4}{x^{3/2}}}=\lim\limits_{x \to \infty} \frac{\sqrt{1+\frac{1}{x^3}}}{\frac{1}{\sqrt{x}}+\frac{4}{\sqrt{x^3}}}=\frac{\sqrt{1}}{0}=\infty \text{, nuevamente indeterminación}
\]

6) 

\[
  \lim\limits_{x \to \infty} \frac{3x-3}{\sqrt{9x^2+7}}=\frac{\infty}{\infty}
\] 

\[
  \lim\limits_{x \to \infty} \frac{\frac{3x}{x}-\frac{3}{x}}{\frac{\sqrt{9x^2+7}}{x}}=\lim\limits_{x \to \infty} \frac{3-\frac{3}{x}}{\frac{\sqrt{9x^2+7}}{\sqrt{x^2}}}=\lim\limits_{x \to \infty} \frac{3-\frac{3}{x}}{\sqrt{\frac{9x^2+7}{x^2}}}= \lim\limits_{x \to \infty} \frac{3 - \frac{3}{x}}{\sqrt{9 + \frac{7}{x^2}}} = \frac{3}{\sqrt{9}} = 1
\] 

7) 

\[
  \lim\limits_{x \to \infty} (\sqrt{2x-1}-\sqrt{x+8})=\infty - \infty
\]  

Multiplicar por el conjugado:

\[
  \lim\limits_{x \to \infty} (\sqrt{2x-1}-\sqrt{x+8}) \cdot \frac{\sqrt{2x-1}+\sqrt{x+8}}{\sqrt{2x-1}+\sqrt{x+8}}=\lim\limits_{x \to \infty} \frac{(\sqrt{2x-1})^2-(\sqrt{x+8})^2}{\sqrt{2x-1}+\sqrt{x+8}}
\]  

\[
  =\lim\limits_{x \to \infty} \frac{x-9}{\sqrt{2x-1}+\sqrt{x+8}} = \frac{\infty}{\infty} \text{, indeterminación}
\]

\[
  = \lim\limits_{x \to \infty} \frac{\frac{x}{x} - \frac{9}{x}}{\frac{\sqrt{2x-1}}{x} + \frac{\sqrt{x+8}}{x}} = \lim\limits_{x \to \infty} \frac{1 - \frac{9}{x}}{\frac{\sqrt{2x-1}}{\sqrt{x^2}} + \frac{\sqrt{x+8}}{\sqrt{x^2}}} = \lim\limits_{x \to \infty} \frac{1 - \frac{9}{x}}{\sqrt{\frac{2x-1}{x^2}} + \sqrt{\frac{x+8}{x^2}}}
\]

\[
  = \lim\limits_{x \to \infty} \frac{1 - \frac{9}{x}}{\sqrt{\frac{2}{x} - \frac{1}{x^2}} + \sqrt{\frac{1}{x} + \frac{8}{x^2}}} = \frac{1}{0}
\]

\section*{Paridad}

\textit{Def.} Una función $f$ se dice que es par si $f(-x)=f(x) \forall x \in Dom_f$.

\textit{Def.} Una función $f$ se dice que es impar si $f(-x)=-f(x) \forall x \in Dom_f$.

\textbf{Exampli gratia:} 

1) $f(x)=x^2$.

$f$ es par. En efecto, sea $x \in Dom_f = \mathbb{R}$.

\[
f(-x)=(-x)^2=x^2=f(x)
\]

\[
  f(-1) = (-1)^2 = 1
\]

\[
  f(1) = (1)^2 = 1
\]

\begin{center}
\begin{tikzpicture}
  \begin{axis}
    [
      axis lines = middle,
      xlabel = $x$,
      ylabel = $y$,
      xmax = 5,
      xmin = -5,
      ymax = 10,
      ymin = -2,
    ]
    \addplot
      [
        domain=-10:10,
        samples=150,
        color=blue,
      ] 
    {x^2};
  \addlegendentry{$x^2$}
  \end{axis}
\end{tikzpicture}
\end{center}

2) $f(x)=x^3$.

\begin{center}
\begin{tikzpicture}
  \begin{axis}
    [
      axis lines = middle,
      xlabel = $x$,
      ylabel = $y$,
      xmax = 5,
      xmin = -5,
      ymax = 10,
      ymin = -10,
    ]
    \addplot
      [
        domain=-5:5,
        samples=150,
        color=blue,
      ] 
    {x^3};
  \addlegendentry{$x^3$}
  \end{axis}
\end{tikzpicture}
\end{center}

$f$ es impar. En efecto, sea $x \in Dom_f = \mathbb{R}$ 

\[
f(-x)=(-x)^3=-x^3=-f(x)
\]

Hay una simetría respecto al eje.

\textit{Funciones pares:} Han de tener simetría con el eje $Y$.

\[
  f(x)=x^4 \, f(x)= x^2-1 \; \; f(x) = \cos(x) \, f(x) = |x|
\]

\textit{ Funciones impares:} Han de pasar en el origen para ser impares. 

\[
  f(x)=\sin(x) \; \; f(x)=mx 
\]

¿Es $f(x)$ par, impar o ninguna? 

\[\text{Sea $f(x)=$} 
  \begin{cases}
    x^2 & , \; x \leq 0\\ 
    x^3 & , \; x > 0 
  \end{cases}
\]

\[
  Dom_f = \mathbb{R}
\]

\begin{center}
\begin{tikzpicture}
  \begin{axis}
    [
      axis lines = middle,
      xlabel = $x$,
      ylabel = $y$,
      xmax = 6,
      xmin = -6,
      ymax = 15,
      ymin = -2,
    ]
    \addplot
      [
        domain=-6:0,
        samples=100,
        color=blue,
      ] 
    {x^2};
  \addlegendentry{$x^2$}
    \addplot
      [
        domain=0:6,
        samples=100,
        color=red,
      ] 
    {x^3};
  \addlegendentry{$x^3$}
  \end{axis}
\end{tikzpicture}
\end{center}

\[
  f(-2) = 4 \text{, $f$ no es par}
\]

\[
  f(2) = 8 \text{, $f$ no es impar}
\]

¿Es $\tan x$ es par, impar o ninguna? 

Como $\tan x = \frac{\sin x}{\cos x}$ y sabemos que $\sin x$ es impar y $\cos x$ es par, entonces, sea $x \in Dom_f$ con $f(x) = \tan x$. 

\[
  Dom_f = \mathbb{R} \setminus \{\frac{(2n + 1) \pi}{2} \, | \, n \in \mathbb{Z}\}
\]

\[
  f(-x) = \tan -x = \frac{\sin -x}{\cos -x} = \frac{-\sin x}{\cos x} = -\frac{\sin x}{\cos x} = -\tan x = -f(x)
\]

\[
  \therefore \text{$\tan x$ es impar}
\]

\end{document}
