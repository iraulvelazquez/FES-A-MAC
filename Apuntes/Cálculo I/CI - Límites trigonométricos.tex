\documentclass{article} % Document type 
\usepackage[spanish]{babel} % Language typesetting 
\usepackage[utf8]{inputenc} % Accented chars without commands
\usepackage[T1]{fontenc} % Accented display corrected
\selectlanguage{spanish} % Default lang, useful for for multi lang switching 
\usepackage{amsmath} % Math 1/3
\usepackage{amssymb} % Math 2/3
\usepackage{stmaryrd} % Math 3/3
\usepackage{tikz} % Crating graphics programmatically
\usepackage{pgfplots} % Extends tikz for 2D and 3D plots for data and math f 
\pgfplotsset{compat=1.18} % ---
\usepackage{circuitikz} % Extends tikz for electrical circuits
\usepackage[dvipsnames, svgnames]{xcolor} % Coloring
\usepackage{soul} % Text highlighting 
\setul{0.5ex}{0.3ex} % ---
\setlength{\parindent}{15pt} % Controls indentantion for the first line of paragraphs 
\setlength{\parskip}{0.8\baselineskip plus 0.2\baselineskip minus 0.1\baselineskip} % Vertical spacing between paragraphs 
\usepackage{geometry} % Layout
\geometry{margin=3cm} % --- 
\usepackage{titlesec} % Global section centering
\titleformat{\section} % --
  {\normalfont\Large\bfseries\centering}{\thesection}{1em}{} % --

\title{Límites trigonométricos}
\date{20-10-2025}
\author{Velázquez Ramírez Carlos Raúl}

\begin{document}
\pagenumbering{gobble}
  \maketitle
  \newpage
  \pagenumbering{roman}

\section*{Límites trigonométricos y tendencia a 0}

Los siguientes 3 casos, y sus extrapolaciones, poseen limites iguales a 1.

\begin{center}
\begin{tikzpicture}
  \begin{axis}
    [
      axis lines = middle,
      xlabel = $x$,
      ylabel = $y$,
      xmax = 25,
      xmin = -25,
      ymax = 1.25,
      ymin = -0.5,
    ]
    \addplot
      [
        domain=-20*pi:20*pi,
        samples=350,
        color=blue,
      ] 
    {(sin(deg(x)))/x};
  \addlegendentry{$\frac{\sin(x)}{x}$}
  \end{axis}
\end{tikzpicture}
\end{center}

\[
  \lim\limits_{x \to 0}\frac{\sin(x)}{x}=1
\]

\[
  \lim\limits_{x \to 0}\frac{x}{\sin(x)}=1
\]

\[
  \lim\limits_{x \to 0}\frac{\sin(kx)}{x}=1
\]

Puesto que el límite de $\frac{\sin(x)}{x}$ y demás son equivalentes a $1$, son útiles para despejar o reducir cietas expresiones trigonométricas más complejas, e.g.,

\[
  \lim\limits_{x \to 0}\frac{\cos(x)-1}{x}=\lim\limits_{x \to 0}\frac{\cos(x)-1}{x}\cdot\frac{\cos(x)+1}{\cos(x)+1}
\]

\[
  =\lim\limits_{x \to 0}\frac{\cos^2(x)-1}{x(\cos(x)+1)}=\lim\limits_{x \to 0}\frac{-\sin^2(x)}{x(\cos(x)+1)}=\lim\limits_{x \to 0}\frac{\sin(x)}{x}\cdot\frac{-\sin(x)}{\cos(x)+1}
\]

\[
  =\lim\limits_{x \to 0}\frac{\sin(x)}{x}\cdot\lim\limits_{x \to 0}\frac{-\sin(x)}{\cos(x)+1}=1\cdot0=0  
\]

\[
  Dom_y=\mathbb{R}\setminus\{0\}
\]

Otro método para atacar límites de funciones trigonométricas es multiplicando por $1$ expresado como el cociente del ángulo trigonométrico respectivo al divisor o dividendo, e.g.,
\[
  \lim\limits_{x \to 0}\frac{2x+5\sin(6x)}{4x+9\sin(3x)}=\frac{0}{0}, \text{ indeterminación}
\]

\[
  \lim\limits_{x \to 0}\frac{\frac{6x}{6x}(2x+5\sin(6x))}{\frac{3x}{3x}(4x+9\sin(3x))}=\lim\limits_{x \to 0}\frac{2x+\frac{3x\sin(6x)}{6x}}{4x+\frac{27x\sin(3x)}{3x}}
\]

\[
  =\lim\limits_{x \to 0}\frac{2x+30x\cdot\frac{\sin(6x)}{6x}}{4x+27x\cdot\frac{\sin(3x)}{3x}}=\lim\limits_{x \to 0}\frac{x(2+30\frac{\sin(6x)}{6x})}{x(4+27\frac{\sin(3x)}{3x})}=\lim\limits_{x \to 0}\frac{2+30\frac{\sin(6x)}{6x}}{4+27\frac{\sin(3x)}{3x}}
\]

\[
  =\frac{\lim\limits_{x \to 0}2+30\lim\limits_{x \to 0}\frac{\sin(6x)}{6x}}{\lim\limits_{x \to 0}4+27\lim\limits_{x \to 0}\frac{\sin(3x)}{3x}}=\frac{2+30\cdot1}{4+27\cdot1}=\frac{32}{31}
\]

\[
  Dom_y=\mathbb{R}\setminus\{0\}
\]

Visualmente, en una forma simple, tenemos que, 

\begin{center}
\begin{tikzpicture}
  \begin{axis}
    [
      axis lines = middle,
      xlabel = $x$,
      ylabel = $y$,
      xmax = 3,
      xmin = -3,
      ymax = 3,
      ymin = -3,
      width=10cm,
      height=10cm
    ]
    \addplot
      [
        domain=-5*pi:5*pi,
        samples=1500,
        color=blue,
      ] 
    {sin(deg(5*x))/sin(deg(9*x))};
  \addlegendentry{$\frac{\sin(5x)}{\sin(9x)}$}
  \end{axis}
\end{tikzpicture}
\end{center}

\[
g(x)=\frac{\sin(5x)}{\sin(9x)}
\]

\[
  Dom_g=\mathbb{R}\setminus\{k\pi|k \in \mathbb{Z}\}
\]

\textit{Nota:} Se ha de considerar que $\lim\limits_{x \to \infty}\sin(x)$ no existe porque sus valores oscilan entre $-1$ y $1$.

\begin{center}
\begin{tikzpicture}
  \begin{axis}
    [
      axis lines = middle,
      xlabel = $x$,
      ylabel = $y$,
      xmax = 10,
      xmin = -10,
      ymax = 5,
      ymin = -5,
    ]
    \addplot
      [
        domain=-5*pi:5*pi,
        samples=100,
        color=blue,
      ] 
    {sin(deg(x))};
  \addlegendentry{$\sin(x)$}
  \end{axis}
\end{tikzpicture}
\end{center}

\section*{Límites infinitos}

\textbf{Exampli gratia:} $\lim\limits_{x \to \infty} \frac{x+3}{x-1}$.

\[
  \lim\limits_{x \to \infty} \frac{x+3}{x-1} = \frac{\infty}{\infty} \text{, indeterminación}
\]

\[
  \lim\limits_{x \to \infty}\frac{\frac{1}{x}(x+3)}{\frac{1}{x}(x-1)}=\lim\limits_{x \to \infty}\frac{1+\frac{3}{0}}{1-\frac{1}{x}}=\frac{1}{1}=1
\]

O, también, 

\[
  \frac{x+3}{x-1} = \frac{x-1+4}{x-1} = \frac{x-1}{x-1} + \frac{4}{x-1} = \frac{4}{x-1} + 1
\]

\begin{center}
\begin{tikzpicture}
  \begin{axis}
    [
      axis lines = middle,
      xlabel = $x$,
      ylabel = $y$,
      xmax = 5,
      xmin = -5,
      ymax = 5,
      ymin = -5,
    ]
    \addplot
      [
        domain=-5:5,
        samples=100,
        color=blue,
       ] 
      {(x+3)/(x-1)};
    \addlegendentry{$\frac{x+3}{x-1}$}
  \end{axis}
\end{tikzpicture}
\end{center}

\section*{Límites con valor absoluto}

\textbf{Exampli gratia:} En el caso $\lim\limits_{x \to 0} \frac{|x-2|-2}{x}$, tenemos que considerar la definición de valor absoluto, 

\[|x-2|= 
  \begin{cases}
    x-2 & \text{si $x \geq 2$}\\ 
    -(x-2) & \text{si $x<2$}
  \end{cases}
\]

Y como el límite tiende a 0, se da uso del segundo caso, 

\[
  \lim\limits_{x \to 0} \frac{|x-2|-2}{x} = \lim\limits_{x \to 0} \frac{-(x-2)-2}{x} = \lim\limits_{x \to 0} \frac{-x+2-2}{x} = \lim\limits_{x \to 0} (-1) = -1
\]

\section*{Vecindades}

\textit{Def.} La suceción $(a_n), \, n \in \mathbb{N}$ converge a $L \in \mathbb{R}$ si $\forall \epsilon > 0 \; \exists N \in \mathbb{N} \, | \, \forall n \geq N \implies |a_n-L| < \epsilon$. 

\textit{Vecindad del punto $a \in \mathbb{R}$ en radio $r$,}

\[
  V_r(a) = ]a-r, a+r[ = \{ x \in \mathbb{R} \, | \, |x-a|<r\}
\] 

\textbf{Exampli gratia:} 

1) Sea $(\frac{1}{n})_{n = 1}^{\infty} = (1, \frac{1}{2}, \frac{1}{3}, \dots)$.

\[
  \lim\limits_{n \to \infty} \frac{1}{n} = 0
\]

Sea $\epsilon > 0$ arbitrario, 

\[
  |a_n - L| = |\frac{1}{n} - 0| = |\frac{1}{n}| = \frac{1}{n}
\]

\[
  \frac{1}{n} < \epsilon \iff 1 < n \epsilon
\]

\[
  \iff n > \frac{1}{\epsilon}
\]

Si $\epsilon = \frac{1}{10}, \; n >10$, 

\[
  a_{n + 1} = \frac{1}{11}
\]

\[
  \text{Acaso } \frac{1}{11} \in V_{\frac{1}{10}} (0) 
\]

\[
  |\frac{1}{11} - 0| < \frac{1}{10}
\]

\[
  \frac{1}{11} < \frac{1}{10} \iff 10 < 11
\]

2) Sea $V_{0.1} (4) = ]3.9, 4.1[ = \{x \in \mathbb{R} \, | \, |x - 4| < 0.1\}$, ¿$x = 4.2 \in V_{0.1} (4)$?

\[
  |4.2 - 4| < 0.1
\]

\[
  0.2 < 0.1 \text{, contradicción}
\]

\[
  \therefore 4.2 \notin V_{0.1} (4)
\]

\end{document}
