\documentclass{article} % Document type 
\usepackage[spanish]{babel} % Language typesetting 
\usepackage[utf8]{inputenc} % Accented chars without commands
\usepackage[T1]{fontenc} % Accented display corrected
\selectlanguage{spanish} % Default lang, useful for for multi lang switching 
\usepackage{amsmath} % Math 1/3
\usepackage{amssymb} % Math 2/3
\usepackage{stmaryrd} % Math 3/3
\usepackage{tikz} % Crating graphics programmatically
\usepackage{pgfplots} % Extends tikz for 2D and 3D plots for data and math f 
\pgfplotsset{compat=1.18} % ---
\usepackage{circuitikz} % Extends tikz for electrical circuits
\usepackage[dvipsnames, svgnames]{xcolor} % Coloring
\usepackage{soul} % Text highlighting 
\setul{0.5ex}{0.3ex} % ---
\setlength{\parindent}{0pt} % Controls indentantion for the first line of paragraphs 
\setlength{\parskip}{0.8\baselineskip plus 0.2\baselineskip minus 0.1\baselineskip} % Vertical spacing between paragraphs 
\usepackage{geometry} % Layout
\geometry{margin=2cm} % --- 

\title{Organización de Computadoras}
\date{25-10-2025}
\author{Velázquez Ramírez Carlos Raúl}

\begin{document}
\pagenumbering{gobble}
  \maketitle
  \newpage
  \pagenumbering{roman}

\section{Representación numérica binaria y punto flotante}

Representación numérica binaria:

\begin{center}
$0000_2=0$\\
$1111_2=15$ 
\end{center}

Codificación binaria con signo (codificación SNNN):

\begin{center}
$0000_2=0$\\
$1111_2=-7$ 
\end{center}

Donde $0=+$ y $1=-$, siendo el bit del extremo izquierdo el bit de signo.

También se tiene que $0000_2=1000_2$. Para arreglar esto, se tiene el \textit{complemento A UNO}. 

Complemento A UNO: 

\[
0000_2=0
\] 
\[
1000_2=-7
\]

Done el bit de signo pasa a ser una función negada del resto de la expresión. 

Complemento A DOS:

\[
0000_2=0
\]
\[
1000_2=-8
\]

Done se asigna $1000_2=-8$ y el resto de números siguen la lógica de A UNO. Aquí es donde se encuentra el overflow. Se puede asignar $1000_2=8$. Como hablamos de hardware, para que esto fucnione todos deben tener la misma asignación para $1000_2$.

\subsection{Suma}

En todas estas codificaciones, la operación de la suma es coherente incluso en la aparición de desborde.

\textit{Nota:} Las computadoras solo saben sumar.
\end{document}
