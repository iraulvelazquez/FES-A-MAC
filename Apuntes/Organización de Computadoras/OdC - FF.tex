\documentclass{article} % Document type 
\usepackage[spanish]{babel} % Language typesetting 
\usepackage[utf8]{inputenc} % Accented chars without commands
\usepackage[T1]{fontenc} % Accented display corrected
\selectlanguage{spanish} % Default lang, useful for for multi lang switching 
\usepackage{amsmath} % Math 1/3
\usepackage{amssymb} % Math 2/3
\usepackage{stmaryrd} % Math 3/3
\usepackage{tikz} % Crating graphics programmatically
\usepackage{pgfplots} % Extends tikz for 2D and 3D plots for data and math f 
\pgfplotsset{compat=1.18} % ---
\usepackage{circuitikz} % Extends tikz for electrical circuits
\usepackage[dvipsnames, svgnames]{xcolor} % Coloring
\usepackage{soul} % Text highlighting 
\setul{0.5ex}{0.3ex} % ---
\setlength{\parindent}{15pt} % Controls indentantion for the first line of paragraphs 
\setlength{\parskip}{0.8\baselineskip plus 0.2\baselineskip minus 0.1\baselineskip} % Vertical spacing between paragraphs 
\usepackage{geometry} % Layout
\geometry{margin=3cm} % --- 

\ctikzset{
  logic ports=ieee,
  logic ports/scale=0.75,
}

\title{Flip-Flops}
\date{04-11-2025}
\author{Velázquez Ramírez Carlos Raúl}

\begin{document}
\pagenumbering{gobble}
  \maketitle
  \newpage
  \pagenumbering{roman}

Circuitos lógicos: 

1) Circuitos combinacionales (independientes del tiempo): CPU, ALU, Unidad de Control decodificadores, etc. \\
2) Circuitos secuenciales (dependientes del tiempo, o estado anterior): RAM, registros, circuitos de reloj, etc. 

\begin{center}
  \begin{circuitikz}
    \draw (0,0) node[and port](AND1){};
    \draw (AND1.in 1) node[not port, scale=0.5, anchor=out](NOT1){};
    \draw (AND1.in 1) -- (NOT1.out);
  \end{circuitikz}
\end{center}

Todo componente en el camino de la corriente reduce su velocidad, desfasando las reacciones en el orden de los nanosegundos. 

\textbf{Flip-Flop:} Reloj que dicta cuando han de ejecutarse laas instucciones. La frecuencia es dada en Hz. en el orden de Megas o Gigas.

\begin{center}
  \begin{circuitikz}[scale=0.75]
    \draw [thick] (0,0) rectangle (2,4);
    \draw (0,1) -- ++(-1,0) node[left]{$Ck$};
    \draw (0,3) -- ++(-1,0) node[left]{$D$};
    \draw (2,1) -- ++(1,0) node[right]{$\neg Q$};
    \draw (2,3) -- ++(1,0) node[right]{$Q$};
    \draw (1,2) node{FF};
  \end{circuitikz}
\end{center}

Donde: 

1) D: Entrada de datos $(0,1)$. \\ 
2) Ck: Entrada reloj. \\
3) Q: Lectura del valor almacenado.

Dar un valor en la entrada $D$ no necesariamente lo almacenará, solo hasta que el valor en la entrada $Ck$ sea $1$.

Operaciones posibles en la memoria RAM: 

1) Lectura. \\ 
2) Escritura. \\ 
3) Nada. 

Latch o cerrojo: 

------------------------------- 



\end{document}
